\documentclass[11pt,twoside,a4paper]{scrartcl}
\usepackage[dvipsnames,svgnames]{xcolor}
\usepackage{ifpdf}
\usepackage{scrpage2}
\usepackage{footmisc}
\ifpdf
  \usepackage[pdfpagemode={UseOutlines},
              pdftitle={Model identifier: "untitled"},
              pdfauthor={Produced by SBML2LaTeX version 1.0beta},
              pdfsubject={SBML model summary},
              pdfkeywords={},
              pdfview={FitBH},
              plainpages={false},
              pdftex,
              colorlinks=true,
              pdfdisplaydoctitle=true,
              linkcolor=royalblue,
              bookmarks,
              bookmarksopen,
              bookmarksnumbered,
              pdfhighlight={/P},
              urlcolor={blue}]{hyperref}
  \usepackage{pdflscape}
  \pdfcompresslevel=9
  \usepackage[pdftex]{graphicx}
\else
  \usepackage[plainpages={false}]{hyperref}
  \usepackage{lscape}
  \usepackage{graphicx}
  \usepackage{breakurl}
\fi
\usepackage{calc}
\usepackage[paper=a4paper,landscape=false,centering]{geometry}
\usepackage{mathptmx}
\usepackage[scaled=.95]{helvet}
\usepackage[english]{babel}
\usepackage[english]{rccol}
\usepackage[version=3]{mhchem}
\usepackage{relsize}
\usepackage{pifont}
\usepackage{textcomp}
\usepackage{longtable}
\usepackage{tabularx}
\usepackage{booktabs}
\usepackage{amsmath}
\usepackage{amsfonts}
\usepackage{amssymb}
\usepackage{mathtools}
\usepackage{ulem}
\usepackage{wasysym}
\usepackage{eurosym}
\usepackage{rotating}
\usepackage{upgreek}
\usepackage{flexisym}
\usepackage{breqn}


\definecolor{royalblue}{cmyk}{.93, .79, 0, 0}
\definecolor{lightgray}{gray}{0.95}
\addtokomafont{sectioning}{\color{royalblue}}
\pagestyle{scrheadings}
\newcommand{\yes}{\parbox[c]{1.3em}{\Large\Square\hspace{-.65em}\ding{51}}}
\newcommand{\no}{\parbox[c]{1.3em}{\Large\Square\hspace{-.62em}--}}
\newcommand{\numero}{N\hspace{-0.075em}\raisebox{0.25em}{\relsize{-2}\b{o}}}
\newcommand{\reaction}[1]{\begin{equation}\ce{#1}\end{equation}}
\newcolumntype{C}[1]{>{\centering\arraybackslash}p{#1}}
\newcommand{\SBMLLaTeX}{{\sffamily\upshape\raisebox{-.35ex}{S\hspace{-.425ex}BML}\hspace{-0.5ex}\begin{rotate}{-17.5}\raisebox{-.1ex}{2}\end{rotate}\hspace{1ex}\LaTeX}}
\cfoot{\textcolor{gray}{Produced by \SBMLLaTeX}}

\subject{SBML Model Report}
\title{Model identifier: ``untitled"}
\date{\today}
\author{\includegraphics[height=3.5ex]{/local/draeger/workspace/DisplaySBML/./resources/SBML2LaTeX}}

\begin{document}
\maketitle
\thispagestyle{scrheadings}


\section{General Overview}
This is a document in SBML Level 2 Version 1 format. Table~\ref{tab:components} provides an overview of the quantities of all components of this model.
\begin{table}[h!]
\centering
\caption{The SBML components in this model.}\label{tab:components}
All components are described in more detail in the following sections.
\begin{tabular}{l|r||l|r}
\toprule
\multicolumn{1}{c}{Element}&\multicolumn{1}{|c||}{Quantity}&\multicolumn{1}{c|}{Element}&\multicolumn{1}{c}{Quantity}\\

\midrule
compartment types&0&compartments&1\\
species types&0&species&5\\
events&0&constraints&0\\
reactions&4&function definitions&0\\
global parameters&0&unit definitions&0\\
rules&0&initial assignments&0\\
\bottomrule\end{tabular}
\end{table}

\section{Unit Definitions}
This is an overview of five unit definitions.
All units are predefined by SBML and not mentioned in the model.
\subsection{Unit \texttt{substance}}
\begin{description}
\item[Notes] Mole is the predefined SBML unit for Mole is the predefined SBML unit for Mole is the predefined SBML unit for 
\item[Definition] $\mathrm{mol}$
\end{description}

\subsection{Unit \texttt{volume}}
\begin{description}
\item[Notes] Litre is the predefined SBML unit for Litre is the predefined SBML unit for Litre is the predefined SBML unit for 
\item[Definition] $\mathrm{l}$
\end{description}

\subsection{Unit \texttt{area}}
\begin{description}
\item[Notes] Square metre is the predefined SBML unit for Square metre is the predefined SBML unit for Square metre is the predefined SBML unit for 
\item[Definition] $\mathrm{m}^{2}$
\end{description}

\subsection{Unit \texttt{length}}
\begin{description}
\item[Notes] Metre is the predefined SBML unit for Metre is the predefined SBML unit for Metre is the predefined SBML unit for 
\item[Definition] $\mathrm{m}$
\end{description}

\subsection{Unit \texttt{time}}
\begin{description}
\item[Notes] Second is the predefined SBML unit for Second is the predefined SBML unit for Second is the predefined SBML unit for 
\item[Definition] $\mathrm{s}$
\end{description}

\section{Compartment}
This model contains one compartment.
\begin{longtable}[h!]{@{}lllC{2cm}llcl@{}}
\caption{Properties of all compartments.}\\
\toprule
Id&Name&SBO&Spatial Dimensions&Size&Unit&Constant&Outside\\
\midrule
\endfirsthead
\toprule
Id&Name&SBO&Spatial Dimensions&Size&Unit&Constant&Outside\\
\midrule
\endhead
\texttt{default}&&&3&1&litre&\yes&\texttt{}\\
\bottomrule\end{longtable}


\subsection{Compartment \texttt{default}}
This is a three-dimensional compartment with a constant size of one\,litre.

\begin{landscape}

\section{Species}
This model contains five species.
Section~\ref{sec:DerivedRateEquations} provides further details and the derived rates of change of each species.
\begin{longtable}[h!]{@{}p{3.5cm}p{6.5cm}p{5cm}p{3cm}C{1.5cm}C{1.5cm}@{}}
\caption{Properties of each species.}\\
\toprule
Id&Name&Compartment&Derived Unit&Constant&Boundary Condition\\
\midrule
\endfirsthead
\toprule
Id&Name&Compartment&Derived Unit&Constant&Boundary Condition\\
\midrule
\endhead
\texttt{s1}&s1&\texttt{default}&$\mathrm{mol}\cdot \mathrm{l}^{-1}$&\no&\no\\
\texttt{s2}&s2&\texttt{default}&$\mathrm{mol}\cdot \mathrm{l}^{-1}$&\no&\no\\
\texttt{s3}&s3&\texttt{default}&$\mathrm{mol}\cdot \mathrm{l}^{-1}$&\no&\no\\
\texttt{s4}&s4&\texttt{default}&$\mathrm{mol}\cdot \mathrm{l}^{-1}$&\no&\no\\
\texttt{s5}&s5&\texttt{default}&$\mathrm{mol}\cdot \mathrm{l}^{-1}$&\no&\no\\
\bottomrule\end{longtable}
\end{landscape}



\begin{landscape}

\section{Reactions}
This model contains four reactions.
 All reactions are listed in the following table and are subsequently described in detail. If a reaction is affected by one or more modifiers, the  identifiers of the modifier species are written above the reaction arrow.
\begin{longtable}[h!]{rp{3cm}p{7cm}p{8cm}p{1.5cm}}
\caption{Overview of all reactions}\\
\toprule
\numero&Id&Name&Reaction Equation&SBO\\
\midrule
\endfirsthead
\toprule
\numero&Id&Name&Reaction Equation&SBO\\
\midrule
\endhead
1&\texttt{re1}&re1&\ce{ $\mathtt{s1}$ +  $\mathtt{s3}$ +  $\mathtt{s4}$ ->  $\mathtt{s2}$ +  $\mathtt{s5}$}&\\
2&\texttt{re2}&re2&\ce{ $\mathtt{s1}$ ->  $\mathtt{s3}$}&\\
3&\texttt{re3}&re3&\ce{ $\mathtt{s3}$ ->  $\mathtt{s4}$}&\\
4&\texttt{re5}&re5&\ce{ $\mathtt{s1}$ ->  $\mathtt{s4}$}&\\
\bottomrule\end{longtable}
\end{landscape}


\subsection{Reaction \texttt{re1}}
This is an irreversible reaction of three reactants forming two products.\begin{description}
\item[Name] re1
\end{description}

\subsubsection*{Reaction equation}
\reaction{ $\mathtt{s1}$ +  $\mathtt{s3}$ +  $\mathtt{s4}$ ->  $\mathtt{s2}$ +  $\mathtt{s5}$}

\subsubsection*{Reactants}
\begin{longtable}[h!]{llc}
\caption{Properties of each reactant.}\\
\toprule
Id & Name & SBO\\
\midrule
\endfirsthead
\toprule
Id & Name & SBO\\
\midrule
\endhead
\texttt{s1}&s1&\\
\texttt{s3}&s3&\\
\texttt{s4}&s4&\\
\bottomrule\end{longtable}

\subsubsection*{Products}
\begin{longtable}[h!]{llc}
\caption{Properties of each product.}\\
\toprule
Id & Name & SBO\\
\midrule
\endfirsthead
\toprule
Id & Name & SBO\\
\midrule
\endhead
\texttt{s2}&s2&\\
\texttt{s5}&s5&\\
\bottomrule\end{longtable}

\subsubsection*{Kinetic Law}

\begin{dmath}
v_{1}=\text{not specified}
\label{v1}
\end{dmath}

\subsection{Reaction \texttt{re2}}
This is an irreversible reaction of one reactant forming one product.\begin{description}
\item[Name] re2
\end{description}

\subsubsection*{Reaction equation}
\reaction{ $\mathtt{s1}$ ->  $\mathtt{s3}$}

\subsubsection*{Reactant}
\begin{longtable}[h!]{llc}
\caption{Properties of each reactant.}\\
\toprule
Id & Name & SBO\\
\midrule
\endfirsthead
\toprule
Id & Name & SBO\\
\midrule
\endhead
\texttt{s1}&s1&\\
\bottomrule\end{longtable}

\subsubsection*{Product}
\begin{longtable}[h!]{llc}
\caption{Properties of each product.}\\
\toprule
Id & Name & SBO\\
\midrule
\endfirsthead
\toprule
Id & Name & SBO\\
\midrule
\endhead
\texttt{s3}&s3&\\
\bottomrule\end{longtable}

\subsubsection*{Kinetic Law}

\begin{dmath}
v_{2}=\text{not specified}
\label{v2}
\end{dmath}

\subsection{Reaction \texttt{re3}}
This is an irreversible reaction of one reactant forming one product.\begin{description}
\item[Name] re3
\end{description}

\subsubsection*{Reaction equation}
\reaction{ $\mathtt{s3}$ ->  $\mathtt{s4}$}

\subsubsection*{Reactant}
\begin{longtable}[h!]{llc}
\caption{Properties of each reactant.}\\
\toprule
Id & Name & SBO\\
\midrule
\endfirsthead
\toprule
Id & Name & SBO\\
\midrule
\endhead
\texttt{s3}&s3&\\
\bottomrule\end{longtable}

\subsubsection*{Product}
\begin{longtable}[h!]{llc}
\caption{Properties of each product.}\\
\toprule
Id & Name & SBO\\
\midrule
\endfirsthead
\toprule
Id & Name & SBO\\
\midrule
\endhead
\texttt{s4}&s4&\\
\bottomrule\end{longtable}

\subsubsection*{Kinetic Law}

\begin{dmath}
v_{3}=\text{not specified}
\label{v3}
\end{dmath}

\subsection{Reaction \texttt{re5}}
This is an irreversible reaction of one reactant forming one product.\begin{description}
\item[Name] re5
\end{description}

\subsubsection*{Reaction equation}
\reaction{ $\mathtt{s1}$ ->  $\mathtt{s4}$}

\subsubsection*{Reactant}
\begin{longtable}[h!]{llc}
\caption{Properties of each reactant.}\\
\toprule
Id & Name & SBO\\
\midrule
\endfirsthead
\toprule
Id & Name & SBO\\
\midrule
\endhead
\texttt{s1}&s1&\\
\bottomrule\end{longtable}

\subsubsection*{Product}
\begin{longtable}[h!]{llc}
\caption{Properties of each product.}\\
\toprule
Id & Name & SBO\\
\midrule
\endfirsthead
\toprule
Id & Name & SBO\\
\midrule
\endhead
\texttt{s4}&s4&\\
\bottomrule\end{longtable}

\subsubsection*{Kinetic Law}

\begin{dmath}
v_{4}=\text{not specified}
\label{v4}
\end{dmath}

\section{Derived Rate Equations}
\label{sec:DerivedRateEquations}
When interpreted as an ordinary differential equation framework, this model implies the following set of equations for the rates of change of each species. 

The identifiers for reactions, which are not defined properly or which are lacking a kinetic equation, are highlighted in \textcolor{red}{red}. 

\subsection{Species \texttt{s1}}
\begin{description}
\item[Name] s1
\item[Initial amount] $0\;\mathrm{mol}$
\end{description}
This species takes part in three reactions (as a reactant in  \hyperref[v1]{\texttt{re1}}, \hyperref[v2]{\texttt{re2}}, \hyperref[v4]{\texttt{re5}}).

\begin{dmath}
\frac{\mathrm d}{\mathrm dt} \mathtt{s1} = -\hyperref[v1]{\textcolor{red}{v_{1}}}-\hyperref[v2]{\textcolor{red}{v_{2}}}-\hyperref[v4]{\textcolor{red}{v_{4}}}
\end{dmath}

\subsection{Species \texttt{s2}}
\begin{description}
\item[Name] s2
\item[Initial amount] $0\;\mathrm{mol}$
\end{description}
This species takes part in one reaction (as a product in  \hyperref[v1]{\texttt{re1}}).

\begin{dmath}
\frac{\mathrm d}{\mathrm dt} \mathtt{s2} = \hyperref[v1]{\textcolor{red}{v_{1}}}
\end{dmath}

\subsection{Species \texttt{s3}}
\begin{description}
\item[Name] s3
\item[Initial amount] $0\;\mathrm{mol}$
\end{description}
This species takes part in three reactions (as a reactant in  \hyperref[v1]{\texttt{re1}}, \hyperref[v3]{\texttt{re3}} and as a product in  \hyperref[v2]{\texttt{re2}}).

\begin{dmath}
\frac{\mathrm d}{\mathrm dt} \mathtt{s3} = \hyperref[v2]{\textcolor{red}{v_{2}}}-\hyperref[v1]{\textcolor{red}{v_{1}}}-\hyperref[v3]{\textcolor{red}{v_{3}}}
\end{dmath}

\subsection{Species \texttt{s4}}
\begin{description}
\item[Name] s4
\item[Initial amount] $0\;\mathrm{mol}$
\end{description}
This species takes part in three reactions (as a reactant in  \hyperref[v1]{\texttt{re1}} and as a product in  \hyperref[v3]{\texttt{re3}}, \hyperref[v4]{\texttt{re5}}).

\begin{dmath}
\frac{\mathrm d}{\mathrm dt} \mathtt{s4} = \hyperref[v3]{\textcolor{red}{v_{3}}} + \hyperref[v4]{\textcolor{red}{v_{4}}}-\hyperref[v1]{\textcolor{red}{v_{1}}}
\end{dmath}

\subsection{Species \texttt{s5}}
\begin{description}
\item[Name] s5
\item[Initial amount] $0\;\mathrm{mol}$
\end{description}
This species takes part in one reaction (as a product in  \hyperref[v1]{\texttt{re1}}).

\begin{dmath}
\frac{\mathrm d}{\mathrm dt} \mathtt{s5} = \hyperref[v1]{\textcolor{red}{v_{1}}}
\end{dmath}

\appendix
\begin{figure}[b!]
\setlength{\fboxrule}{.1cm}
\setlength{\fboxsep}{.3cm}
\fcolorbox{lightgray}{white}{\begin{minipage}{.945\textwidth}
% \renewcommand{\footnoterule}{}% \renewcommand{\thempfootnote}{\fnsymbol{mpfootnote}}
\footnotesize\SBMLLaTeX{} was developed by Andreas Dr\"ager\footnote{Center for Bioinformatics T\"ubingen (ZBIT), Germany}, Hannes Planatscher\mpfootnotemark[1], Dieudonn\'e M Wouamba\mpfootnotemark[1], Adrian Schr\"oder\mpfootnotemark[1], Michael Hucka\footnote{California Institute of Technology, Beckman Institute BNMC, Pasadena, United States of America}, Lukas Endler\footnote{European Bioinformatics Institute, Wellcome Trust Genome Campus, Hinxton, United Kingdom}, Martin Golebiewski\footnote{EML Research gGmbH, Heidelberg, Germany}, Wolfgang M{\"u}ller\mpfootnotemark[4], and Andreas Zell\mpfootnotemark[1]. Please see \url{http://www.ra.cs.uni-tuebingen.de/software/SBML2LaTeX} for more information.
\end{minipage}}
\end{figure}
\end{document}
