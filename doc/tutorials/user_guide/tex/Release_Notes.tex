\chapter{Release Notes}

This chapter gives a brief overview of the main user-visible changes of
SBMLsqueezer since its first release. As such, this chapter gives you an
historic overview about the development of SBMLsqueezer as a large software
project.  

\section{Version 1.0}

Release date: April 28\textsuperscript{th} 2008.

This is the initial release of the software that has been made available as
a supplement to the original publication from in \citeyear{Draeger2008}. All
features of this version are therefore described in the corresponding
publication \cite{Draeger2008}.

\section{Version 1.1}

Release date: February 9\textsuperscript{th} 2009.

\subsection{New features}

\begin{dinglist}{52}
\itemcolor{blue}
\item SBMLsqueezer now allows you to specify whether new parameters should be stored
locally (only valid within the respective kinetic law) or globally (valid for
the whole model). Both, the main window and the context menu contain switches to
make this decision. Please note that the $k_\mathrm{G}$ parameters of the
thermodynamically independent convenience kinetics are always stored globally
because these do not belong to a specific reaction but to a specific reacting
species.

\item The context menu of SBMLsqueezer now also shows already existing kinetic
equations and renders their formula. The name of the kinetic equation created by
SBMLsqueezer is now written to the notes element of the newly created kinetic
law.

\item The \LaTeX{} export function was improved:
\begin{itemize}
  \item Several choices allow for a customization of the output.
  \item The context menu also allows you to export a LaTeX file for a specific
        kinetic equation.
  \item It now includes HTML2\LaTeX{} that converts notes elements from \XHTML to
        \LaTeX. Therefore, SBMLsqueezer also includes your notes in its model
        report.
\end{itemize}

\item SBMLsqueezer version~1.1 is distributed under the terms of the GNU General
\href{http://www.gnu.org/licenses/gpl.html}{Public License (GPL)}.
\end{dinglist}

Please note that SBMLsqueezer's \LaTeX{} export function is, however, still not
comprehensive. For a fully-featured model report generator, we strongly recommend
to have a look on our latest project: \SBMLLaTeX on
\url{http://webservices.cs.uni-tuebingen.de} and
\url{http://www.cogsys.cs.uni-tuebingen.de/software/SBML2LaTeX}.

\subsection{Bug fixes}

\begin{dinglist}{54}
\item In some cases SBMLsqueezer's context menu did not offer the full list of
available and applicable kinetic equations for certain reactions.

\item If the number of products was higher than the number of reactants a null pointer
exception was thrown in the thermodynamically independent form of the
convenience kinetics due to an incorrect access to the list of products of a
reaction.

\item In the thermodynamically independent convenience kinetics the root function was
called with only one argument. This is actually not incorrect but earlier
versions of \libSBML are unable to interpret such an implicit square root and
require a 2 as the second argument.

\item In the thermodynamically independent convenience kinetics the exponent of the
anabolic term and the catabolic term were incorrect if the stoichiometry was not
equal to one. In this case, a division of the exponent by two was missing.

\item In the mass action kinetics a copy and paste error in the source code lead to
incorrect pre-factors for activation and inhibition if both effects were
assigned to the same reaction. Instead of receiving one activation function and
one inhibition function there was the same activation twice.

\item For several reasons null pointer exceptions could occur when trying to create
\LaTeX{} code from a model.
\end{dinglist}


\section{Version 1.2}

Release date: July 31\textsuperscript{st} 2009.

\subsection{New features}

\begin{dinglist}{52}
\itemcolor{blue}
\item SBMLsqueezer now automatically checks for updates. If a newer version of
SBMLsqueezer is available on-line, the user is notified by a message window in
the bottom right corner of the screen. Furthermore, the release notes of the
most recent SBMLsqueezer version are shown to the user by clicking on \keys{show
release notes}.

\item In contrast to previous versions, SBMLsqueezer now indicates if a model already
contains rate equations for all reactions or no reactions at all. In earlier
versions SBMLsqueezer did not overwrite existing kinetic equations and did also
not state why.

\item SBMLsqueezer's complete internal data structure was improved: It has become much
more efficient and simpler.
\end{dinglist}

\subsection{Improved Adaptation of SBMLsqueezer to \CellDesigner 4.0.1}

Due to the changes in \CellDesigner, SBMLsqueezer only offered zeroth order mass
action kinetics as available option to model transcription and translation
processes. The reason for this was that the arrows for
transcriptional/translational activation disappeared. These effects have now to
be covered by trigger and physical stimulation. Unfortunately, \CellDesigner maps
the old arrows for transcriptional/translational activation to catalysis instead
of ``trigger''. Therefore, we also have to cover the case of having a
transcription or translation ``catalyzed'' by some stimulator. This has now been
implemented and SBMLsqueezer therefore offers several kinetic rate equations for
both processes. The same holds true for the ``batch'' mode of SBMLsqueezer:
besides transcriptional/translational activators also all kinds of catalysts and
activators/inhibitors are accepted for the Hill equation.

We now apply the inhibition pre-factor from convenience kinetics also to
reversible Michaelis-Menten kinetics with multiple inhibitors. In cases where
the formulas of Michaelis-Menten and convenience kinetics are equal, only one of
both is offered to the user:
\begin{itemize}
\item For reversible or irreversible uni-uni reactions without inhibition or
      activation only the Micha\-elis-Menten equation is selectable and
\item in the case of reversible uni-uni reactions with multiple inhibitors only
      convenience kinetics can be applied.
\item However, if the stoichiometric matrix of the reaction system does not have
      a full column rank, the thermodynamically independent form of the
      convenience kinetics can be selected as an alternative to the
      Michaelis-Menten equation.
\end{itemize}
Additionally, the activation pre-factor is now also applied for the mixed-type
inhibition of irreversible enzymes by mutually exclusive inhibitors
(\href{identifiers.org/biomodels.sbo/SBO:0000275}{SBO:0000275}) if activators
are assigned to the reaction.

The new \SBGN representation of transcription and translation uses the trigger
symbol (for the gene or the \RNA molecule, depending on the process) and a
reaction from some source to \RNA or from some source to a protein \citep{LeNovere2009}.
SBMLsqueezer did in these cases not suggest the Hill equation as a possible rate law because
it still required the translation/transcription arrows that are going to be
deprecated. Since this version, SBMLsqueezer also produces the identical form of
the Hill equation for the old style of transcription/translation reaction and
the new \SBGN-compliant form.

\subsection{Bug fixes}

\begin{dinglist}{54}
\item Local parameters could not be removed completely if not necessary (this happened
only in SBMLsqueezer version~1.1 because SBMLsqueezer version~1.0 did not create
or delete local parameters at all).

\item In Michaelis-Menten kinetics activators were multiplied incorrectly (only in the
preview, not in the equation itself)

\item The help browser could not be started in SBMLsqueezer version 1.1 because of a
null pointer exception (an image could not be loaded correctly).

\item Problems of the ``batch mode'':
\begin{itemize}
\item In some cases SBMLsqueezer did not show the correct SBO numbers (for 
      ``Hill equation'' and ``Henri-Michaelis-Menten equation'').
\item In the kinetics summary table SBMLsqueezer did not list the $k_G$
      parameters of the thermodynamically independent convenience kinetics in
      the column ``parameters''.
\item Sometimes there was a problem with the equation preview.
\end{itemize}
\end{dinglist}

\section{Version 1.2.1}

Release date: August 7\textsuperscript{th} 2009.

\subsection{New features}

\begin{dinglist}{52}
\itemcolor{blue}
\item In the context menu, SBMLsqueezer now remembers the rate law when switching the
reaction from reversible to irreversible and automatically selects the
corresponding equation if available.
\end{dinglist}

\subsection{Bug fixes}

\begin{dinglist}{54}
\item In the Hill equation instead of inhibitors SBMLsqueezer accessed the list of
activators to create an inhibition term (only in SBMLsqueezer~1.2).

\item When trying to remove unnecessary parameters from the model, SBMLsqueezer~1.2
deleted parameters, whose \acp{ID} contain upper case letters, no matter if
these occur in kinetic equations because of an incorrect String comparison.

\item The thermodynamically independent convenience kinetics was not created correctly
in SBMLsqueezer~1.2: the integer one was subtracted in the denominator in cases
where this was incorrect and the parameter $k_\mathrm{V}$ was not multiplied with the
formula.
\end{dinglist}

\section{Version 1.3}

Release date: April 2\textsuperscript{nd} 2010.

Parts of the content in these release notes have been taken from the Ph.D.
thesis of \citealp{Draeger2011a}, where this section is much more elaborated.

\subsection{New features}

\subsubsection{A new and purely \Java-based \SBML data structure}

SBMLsqueezer's data structure was completely changed: Earlier versions were
based on String concatenation to create kinetic formulas with all participating
species and parameters. Now SBMLsqueezer is based on an abstract syntax tree
representation of the formula and also creates Parameter objects rather than
simply their \ac{ID}.

This new data structure is actually an almost complete \Java implementation of
\SBML and has therefore become a separate project, \JSBML, available at
\url{http://sourceforge.net/projects/jsbml}. On the \SBML homepage, you can find
a separate mailing list about this project: \url{http://sbml.org/Forums}

With \JSBML at hand, SBMLsqueezer does no longer manipulate \CellDesigner's
plug-in Objects directly. If you start SBMLsqueezer from \CellDesigner's plug-in
menu or its reaction context menu, SBMLsqueezer will copy all required parts of
the model into \JSBML objects. Only this copy of the model is manipulated. By
clicking on \keys{OK} or \keys{Apply}, all changes will be synchronized with \CellDesigner's
original data structures

\subsubsection{Stand-Alone Mode}

SBMLsqueezer now also runs in a stand-alone mode. With the help of \libSBML,
valid \SBML files can be read and copied into the \JSBML data objects. \JSBML now
mirrors the complete definition of \SBML up to Level 2 Version 4. As in the
\CellDesigner plug-in mode, SBMLsqueezer only manipulates \JSBML objects. By
clicking on \keys{Save}, changes are written into the \SBML file.

Command-line mode: All features of SBMLsqueezer are now also available from the
command line. This allows users to write shell or batch scripts that access
SBMLsqueezer's functions.

The \API of SBMLsqueezer has been simplified.
There is now just one function, squeeze in the main class SBMLsqueezer that can
be called to read in a model from a valid \SBML file and to write the result,
i.e., an \SBML model including new \texttt{KineticLaw}s, new \texttt{Parameter}s, new or adapted
\texttt{UnitDefinition}s, into the given out file. Before creating kinetic formulas for
the model, the user probably wants to adjust SBMLsqueezer's configuration.
To this end, the methods\texttt{set(CfgKey, <boolean|double|int|String>)} in class
\texttt{SBMLsqueezer} allow the programmer to set any configuration key and value pair,
similar to the command-line mode. Here we give a minimal example of how to use
this new \API:
\begin{lstlisting}[language=Java, caption={Usage of SBMLsqueezer 1.3 via its \acs{API}}, label={lst:1.3APIUsage}]
public static void main(String[] args) {
  // Initialize SBMLsqueezer with appropriate SBML readers/writers
  SBMLsqueezer squeezer = new SBMLsqueezer(new LibSBMLReader(), new LibSBMLWriter());
  // Configure SBMLsqueezer
  squeezer.set(CfgKeys.OPT_ALL_REACTIONS_ARE_ENZYME_CATALYZED, true);
  squeezer.set(CfgKeys.OPT_DEFAULT_COMPARTMENT_INITIAL_SIZE, 1.0);
  squeezer.set(CfgKeys.POSSIBLE_ENZYME_RNA, true);
  squeezer.set(CfgKeys.KINETICS_UNI_UNI_TYPE, MichaelisMenten.class.getName());
  squeezer.set(CfgKeys.KINETICS_OTHER_ENZYME_REACTIONS, ConvenienceKinetics.class.getName());
  try {
    // Create kinetic equations, parameters, units etc. and save the result
    // args contains infile and outfile path
    squeezer.squeeze(args[0], args[1]);
  } catch (Throwable e) {
    e.printStackTrace();
  }
}
\end{lstlisting}
Since SBMLsqueezer still requires an installation of \libSBML on the user's
system, which can sometimes become problematic, a more convenient way of using
SBMLsqueezer has now been made available: the SBMLsqueezer web application does
not require any local installation and is freely available at
\url{http://webservices.cs.uni-tuebingen.de}. In this framework, work-flows can
be created in which SBMLsqueezer can directly be linked to the full version of
\SBMLLaTeX. The full version of \SBMLLaTeX contains many additional features
that are not included in SBMLsqueezer's version of \SBMLLaTeX. Hence, it is
highly recommended to make use of the on-line version of \SBMLLaTeX.


\subsubsection{Configuration}

SBMLsqueezer now memorizes every setting and changed parameters from the
\GUI whether in \CellDesigner plug-in or stand-alone
mode. In the stand-alone case, SBMLsqueezer even memorizes given command-line
arguments. These settings are read from a configuration file at the next start.
This is especially convenient for open and save directories. At any time, the
user can change all settings to the defaults by clicking on the designated button
in the preferences dialog.

\subsubsection{Improved Systems Biology Ontology support}

SBMLsqueezer is now fully based on the \acl{SBO} (\SBO). The \JSBML
copy of \CellDesigner's plug-in data structures maps all \CellDesigner-specific
annotations to corresponding \SBO terms. Afterwards these are mapped back. \SBO
term ``enzymatic catalyst''
(\href{identifiers.org/biomodels.sbo/SBO:0000460}{SBO:0000460}) was created
because it is required by SBMLsqueezer to distinguish enzymes and other
catalysts, such as certain inorganic ions. SBMLsqueezer's stand-alone version
interprets \SBO terms that can be found in the model and changes the annotation
of ModifierSpeciesReferences from ``catalyst''
(\href{identifiers.org/biomodels.sbo/SBO:0000013}{SBO:0000013}) to ``enzymatic
catalyst'' if the SBO term of the corresponding Species is one of those
belonging to the family of materials considered as enzymes (SBMLsqueezer still
offers check boxes to indicate certain kinds of species as enzymes).

Earlier versions of SBMLsqueezer indicated the \SBO term \acp{ID} of newly
created kinetic equations if these were already defined in \SBO. However, this
was more or less just for information to the user and could neither be saved nor
be considered during any other processes. This version of SBMLsqueezer is able
to not only take \SBO terms into account while creating rate laws, but can even
save \SBO terms in the resulting \SBML file. However, in \CellDesigner plug-in
mode, SBMLsqueezer is not yet able to save \SBO term \acp{ID} in \CellDesigner data
objects due to the missing functionality in its \API.

SBMLsqueezer now contains an own parser to identify the correct \SBO term \acp{ID} and
attributes that are required to annotate all newly created objects.

Whenever possible, SBMLsqueezer annotates newly created objects, such as
\texttt{Parameter}s or \texttt{UnitDefinition}s, with \SBO \acp{ID}.

\subsubsection{\MIRIAM Support}

In contrast to earlier versions, SBMLsqueezer now also interprets and uses
\MIRIAM tags. The new command-line and configuration key option
\verb!OPT_IGNORE_THESE_SPECIES_WHEN_CREATING_LAWS! takes a list of
comma-separated \MIRIAM \acp{ID} as value. Species annotated with
\verb!BQB_IS! and the given \MIRIAM resource are then ignored when creating rate
equations. This is, for instance, useful if small molecules such as water take
part in a reaction. The concentration of water can hardly be measured and it is
assumed to occur abundantly within the cell. By default, the list of species
whose influence is to be neglected when creating new kinetic equations contains
the \KEGG compound \acp{ID} for the substances listed in
\vref{tab:MIRIAMignoreList}.

\subsubsection{New kinetic equations and extensibility}

In earlier versions the inclusion of new kinetic equations into SBMLsqueezer was
not an easy task. After writing a class that actually generates the new rate
law, a rule had to be implemented of when to make this rate law available.
Furthermore, the GUI had to be changed to include this new formula. Summarized,
several changes in the program code were necessary to include additional rate
laws.

Now SBMLsqueezer is based on \Java reflection, i.e., initially SBMLsqueezer does
not know any one of its available kinetic equations. These are all loaded when
initializing SBMLsqueezer. Currently, twelve interfaces define the properties of
all kinetic equations. The rules of when select which kind of rate equation only
make use of the implemented interfaces. Hence, new kinetic equations can be
incorporated into SBMLsqueezer simply by implementing designated interfaces.
Since SBMLsqueezer is an open-source project, users can customize it easily and
extend it with desired rate laws.

Much effort was put into the extension of SBMLsqueezer with additional rate laws
for gene-regulatory processes. Earlier versions of SBMLsqueezer only provided
the Hill equation variant suggested by \citet{Hinze2007}. This equation has now
been renamed to Hill-Hinze equation to highlight that it is actually a
modification of the original rate law. Only in special cases this equation
equals the traditional Hill equation (SBMLsqueezer will highlight these cases).
Now nine additional rate equations were included:
\begin{description}
\item[Additive Model Linear] A generalized super class of all other additive rate laws for gene-regulatory processes.
\item[Additive Model Non-Linear] defined in the paper ``The Net\emph{Gene}rator Algorithm] Reconstruction of Gene Regulatory Networks'' by \citealp*{Toepfer2007}
\item[H-Systems] \citealp*{Spieth2006} ``Comparing Mathematical Models on the Problem of Network Inference''.
\item[NetGenerator Linear Model] ``The Net\emph{Gene}rator Algorithm: Reconstruction of Gene Regulatory Networks'' by \citealp*{Toepfer2007}
\item[NetGenerator Non-Linear Model] ``The Net\emph{Gene}rator Algorithm] Reconstruction of Gene Regulatory Networks'' by \citealp*{Toepfer2007}
\item[S-Systems] \citet{Tournier2005}: ``Approximation of dynamical systems using S-systems theory: application to biological systems''
\item[Hill-Radde equation] ``Modeling Non-Linear Dynamic Phenomena in Biochemical Networks`` by \citealp*{Radde2007}
\item[Vohradsky] ``Neural network model of gene expression.'' of \citealp{Vu2007}
\item[Weaver] ``Modeling regulatory networks with weight matrices'' by \citealp*{Weaver1999}
\end{description}

The following new generalized kinetic equations for metabolic reactions were
also implemented in SBMLsqueezer, which are all defined by \citet{Liebermeister2010}:
\begin{itemize}
\item Common modular rate law (CM)
\item Direct binding modular rate law (DM)
\item Force-dependent modular rate law (FM)
\item Power-law modular rate law (PM)
\item Simultaneous binding modular rate law (SM).
\end{itemize}
Each one of these equations is only available in a reversible mode and the user
must choose one of the three possible versions that determine the degree of
thermodynamic correctness. The \emph{weg} version is the most complicated form but
always thermodynamically correct. The \emph{hal} version ensures thermodynamic
correctness in most cases, whereas for the \emph{cat} version thermodynamic properties
cannot be guaranteed to be correct. The versions can be selected in the
preferences dialog or provides as a command-line option.

Finally, the generalized form of Hill's equation as proposed by \citet[in
``Fundamentals of Enzyme Kinetics'', p. 314]{Cornish-Bowden2004} has been
implemented and is now also available for metabolic reactions.

If the corresponding flag is set, boundary conditions are also set for species
that represent empty sets, i.e., degraded species, sources, or sinks in a
reaction.


\subsubsection{Support and assignment of units and unit definitions}

For multi-compartment models concentration versus molecule counts matters.
SBMLsqueezer now provides two different ways of ensuring unit correctness.
First, each species can be brought to units of concentration. To this end,
SBMLsqueezer interprets the \texttt{hasOnlySubstanceUnits} attribute of each species.
Depending on whether this is set to true, the species is divided by the size of
the surrounding compartment. Otherwise the species already represents a
concentration and nothing is done. Second, all species can be brought to
molecule count units. This is again achieved with the help of the
\texttt{hasOnlySubstanceUnits} attribute, but now SBMLsqueezer multiplies with the
compartment size if this attribute is false.

In contrast to earlier versions, SBMLsqueezer now equips all newly created
Parameter objects with units. To this end, SBMLsqueezer derives the correct
units from the context. For instance, in the mass action kinetics the units of
the catalytic constants depend on the order of the reaction and have therefore
to be computed whenever such a rate law is created. Furthermore, SBMLsqueezer
also to considers the \texttt{hasOnlySubstanceUnits} property of each species and the way
in which the surrounding compartment comes into play to derive the units of
parameters.

To make models more realistic, SBMLsqueezer redefines the pre-defined \SBML unit
definition substance from mol to mmol and volume from l to ml.

If possible, SBMLsqueezer avoids creating new \texttt{UnitDefinition} objects. It first
tries to find equivalent and already existing definitions in the model.

\subsection{Bug Fixes}

\begin{dinglist}{54}
\item In the original paper of \citet{Hinze2007}, inhibition was expressed with 1 -
product of all inhibition functions. Earlier versions of SBMLsqueezer created a
product that run over all inhibitors with (1 - inhibition function) for each
such function.

\item In all gene-regulation kinetics, it is now assumed that degradation processes
are distinct reactions that can be, e.g., modeled using a mass action rate law.
Hence, no degradation terms can be found in gene-regulation kinetics anymore.

\item The rate law corresponding to
\href{identifiers.org/biomodels.sbo/SBO:0000266}{SBO:0000266} was not correct.
Earlier versions of SBMLsqueezer produced the fraction $K_{\mathrm{M}js}/K_{\mathrm{ib}j}\cdot I$, but
correct is $S\cdot\nicefrac{I}{K_{\mathrm{ib}j}}$ (see Equation 10 in the supplementary material ``Kinetic
Laws''; the correct version of this equation can be found at
\url{http://www.ebi.ac.uk/sbo/main/SBO:0000266}).

\item In case of an explicit enzyme catalysis, earlier versions of SBMLsqueezer did not
multiply the numerator of equation
\href{identifiers.org/biomodels.sbo/SBO:0000273}{SBO:0000273} with the enzyme.
Therefore, it always produced the $V_\mathrm{max}$ version of the equation, but the parameter
was called $k_\mathrm{cat}$.
\end{dinglist}

\section{Version 2.0}

Many concepts that have been introduced with version~1.3 have been revised and
updated. The entire handling of user settings has been massively changed and now
uses the concept of user preferences instead of configuration files. The source
code has been cleaned and re-factored in many positions and lots of development
time has been invested into with the library \JSBML.

\subsection{New features}

SBMLsqueezer can now be used in several different ways:
\begin{itemize}
\item As fully-featured stand-alone program without the need to install any
      further software, based on the official \JSBML version  1.0$\upbeta$1
\item As a stand-alone program based on a \libSBML back-end, which means that
      real off-line \SBML validation is possible.
\item Irrespective of whether \JSBML or \libSBML is used as \SBML library, the
      stand-alone version provides both, a \GUI as well as
      a fully-featured command-line version. For large-scale rate law
      generation, this command-line interface has already proven to be useful
      (as part of the path2models project).
\item SBMLsqueezer can be launched directly from the
      \href{http://www.cogsys.cs.uni-tuebingen.de/software/SBMLsqueezer/}{project's website}
      as a \JavaWebStart program without the need of any local installation.
\item As a plug-in for \CellDesigner 4.3
\item As a \Garuda gadget, as which it can communicate with further Garuda
      gadgets
\end{itemize}
Hence, the usage of the program has been greatly simplified.
Further new program features comprise:
\begin{dinglist}{52}
\itemcolor{blue}
\item SBMLsqueezer fully supports all Levels and Versions of \SBML up to the most
      recent specification Level 3 Version 1.
\item A direct query wizard for \SABIO has been implemented and allows users to
      extract experimentally obtained rate laws from this database together with
      parameters, values, and units. The search engine provides several settings for
      experimental conditions.
\item SBMLsqueezer includes now several \JUnit test cases to check if the generated
      kinetic equations are correct.
\item It includes now a full version of \SBMLLaTeX and can directly open \PDF files
      that are generated to document the content of the model.
\item The renderer for equations and formula has been changed and provides more
      capabilities. SBMLsqueezer windows and context menus are now zoom-able and
      re-sizable.
\item Improved user preferences menu and management for each individual \OS.
\item The user interface (\GUI and command-line) is fully bilingual (\German and \English), and partially also
      translated to \Chinese, but models and their labels are only created in English for the sake
      of a better international re-usability.
\item A new search function allows users to query the model data structure in the
      \GUI.
\item Lots of effort has been put to better adapt SBMLsqueezer to individual
      platforms. The support for \MacOSX has been greatly improved, the program now
      provides many features of native applications for \MacOSX.
\item The documentation of the \API has been improved.
\end{dinglist}

\subsection{Bug fixes}

The communication between SBMLsqueezer and \CellDesigner was very inefficient.
The new plug-in interface in \JSBML provides a much better performance.
Errors in kinetic equations and units of parameters have been detected and
solved:
\begin{dinglist}{54}
\item Sometimes SBMLsqueezer created invalid \acp{ID} for newly introduced
      units.
\item The units of the association and dissociation constant in generalized mass
      action kinetics could not be derived correctly if a catalyst interfered
      with the reaction, because the units of the catalyst were not taken into
      account.
\item The units of the half saturation constant in the (generalized) Hill
      equation were not correctly derived. The derivation of units of kinetic
      parameters has been improved.
\end{dinglist}  
Several minor issues with the \GUI have been solved.

\subsection{Known Issues}

In \SBML models, it should be possible to evaluate all kinetic equations to
extend of reaction units per time units. Usually, the extend of a reaction would
be a variant of a substance unit. However, it has been recognized that some
kinetic equations cannot under all conditions be evaluated to extend per time
units. The reasons are often the structure of the definition of the rate law.
Some equations that have been defined for gene-regulatory networks, for
instance, give a purely phenomenological description of the system (e.g.,
S- or H-systems). In other equations, it is assumed that the equilibrium constant
is always dimensionless. If this is not the case, it cannot be guaranteed that
SBMLsqueezer will create correct units for these equations (e.g., modular rate
laws and convenience kinetics). It is therefore recommended to check the units
of generated rate laws, even if in the vast majority of cases these will be
correct. Please note that in this version the general unit handling has been
extensively revised and improved and is in the vast majority of cases very
reliable. SBMLsqueezer displays the derived units of equations, so that
problematic equations can be easily identified.
