% 01_Introduction
\chapter{Overview}

The program SBMLsqueezer is a versatile generator for complex kinetic equations
of reactions and processes in biochemical networks \citep{Draeger2008,
Draeger2010a, Draeger2011a}. The kinds of equations supported by the program are
numerous, including traditional approaches such as the (Henry-) Michaelis-Menten
equation \citep{Michaelis1913} for uni-uni enzyme reactions, or very recent
approaches, e.g., the modular rate laws \citep{Liebermeister2010}. It provides
equations for gene-regulatory processes, such as the Hill-Hintze equation
\citep{Hinze2007}.
\nocite{Savageau1969, Segel1993, Heinrich1996, Weaver1999, Bisswanger2000,
Cornish-Bowden2004, Liebermeister2006, Toepfer2007, Vu2007, Radde2007a}

The data format understood by SBMLsqueezer is the Systems Biology Markup
Language (SBML) in all of its levels and versions \citep{Hucka2001, Hucka2003,
M.Hucka03012003, Hucka2007, Hucka2008, Hucka2010a, Finney2003, Finney2006}.
All generated equations are stored, depending on the version of SBML, in form of
infix formula strings (SBML Level~1) or MathML \citep{Buswell1999} expressions
(for all other SBML versions).

Kinetic equations contain a set of parameters, whose units must be derived to
ensure that the overal equation can be interpreted in extent per time units.
Users of SBMLsqueezer don't have to care much about units because the program
does this for you. The program also annotates all created equations and
parameters with appropriate terms from the Systems Biology Ontology and also
with MIRIAM-compliant controlled vocabulary terms (Minimal Information Required
In the Annotation of Models), where appropriate \citep{Le2005, Novere2006b,
Laible2007, Courtot2011}.

The selection of the kinetic equations is done automatically by the program,
where several settings allow users to influence the algorithm's choice.
In many cases, SBMLsqueezer suggests multiple equations for the same process,
from which the user can choose as desired. This behavior of the program is
important to ensure that only appropriate equations can be selected at any time.

In order to decide, which equations can be applied to a reaction of interest,
SBMLsqueezer analyzes several properties, such as the number and type of
reactants, products, and modifiers, if the reaction is reversible, etc.
Here, reaction of interest means that the user can either select individual
reactions from a larger model and equip these with kinetic equations, or rate
equations can be created in one go for the entire model.
Thereby, already existing equations can either be kept or overwritten, depending
on your choice.
You can influence and change all choices of the program, even in the one-go-mode.

If you are connected to the internet, you can also use SBMLsqueezer to extract
experimentally determined kinetic equations from the rate law database SABIO-RK
\citep{Wittig2006, Rojas2007, Krebs2007, Wittig2012}.
To this end, SBMLsqueezer provides several settings, such as the organism,
temperature, or pH value under which the reaction's rate was determined. Again,
you can extract equations from SABIO-RK for an entire network in one go, or
individually select reactions of interest. 

The program itself can be customized and used in several ways. It remembers your
last opened files, the window's size, your rate law selection and so forth.
You can run it as a stand-alone tool, or as a plug-in for the well-known program
CellDesigner \citep{Funahashi2003, Funahashi2006, Funahashi2007a,
Funahashi2008}.
By default, the stand-alone version is based on the JSBML backend 
\citep{Draeger2011b}, but you can also use libSBML \citep{Bornstein2008} as its
SBML backend instead if its more elaborated offline model validation matters.
If you neither like to download SBMLsqueezer, nor to install any software on
your local computer, you can even use SBMLsqueezer as a Galaxy-based webservice,
available at \url{http://webservices.cs.uni-tuebingen.de/}. 

Finally, the command-line mode of SBMLsqueezer provides all capabilities of the
program without any restriction.
Furhtermore, the fully accessible application programming interface can be used
to integrate SBMLsqueezer as an equation generating core into your end-user
program.
You can hence equip a large number of files with kinetic equations without the
need to open each file in a graphical user interface. The usefulness of this
approach has recently been demonstrated as part of the path2models project, in
which more than 142,000 SBML models have been processed with SBMLsqueezer.

For the documentation of your model, SBMLsqueezer includes the program
\SBMLLaTeX, which generates a comprehensive report of your model, including a
detailed description of all components and equations \citep{Draeger2009b,
Draeger2010a}.
These reports can be very handy to support scientific writing, because you can
easily copy the formulas into your paper.


%%%%%%%%%%%%%%%%%%%%%%%%%%%%%%%%%%
% 02_Installation
%%%%%%%%%%%%%%%%%%%%%%%%%%%%%%%%%%
\chapter{Installation}

TODO: This is some template text. Please modify / replace it with your own.


\section{Requirements}
\subsection{Software}

InCroMAP is entirely written in Java\TTra and runs on any operating system
where a suitable Java Virtual Machine (JDK version 1.6 or newer) is installed.
See, for example, the Java SE download
page\footnote{\url{http://www.oracle.com/technetwork/java/javase/downloads/index.html}\label{fn:jvmldl}}.

\subsection{Hardware}

With at least 1\,GB main memory, you should be able to perform most tasks
without any problem. For large datasets, you should have at least 2\,GB of main
memory. \newline An active internet connection is required for most operations.

\subsection{Installation as a plug-in for CellDesigner}

\subsection{Using the libSBML backend}

\subsection{Integration into GARUDA}

\section{Starting the application}
\label{startingTheProgram}

If you downloaded a ZIP-file, you need to unzip it before starting the
application. Depending on your operating system, you should use the provided
shell scripts for starting the application. This is \texttt{start.sh} for Linux
or \texttt{start.bat} for Windows. On MAC OS, you have to create your own
shortcut. You can start the application on all operating systems by typing

\begin{lstlisting}[language=bash,numbers=none]
java -jar -Xms128m -Xmx1024m InCroMAP.jar
\end{lstlisting}

\noindent on your command prompt. Please note that you might have to change
\texttt{InCroMAP.jar} for the real name of the JAR-file, e.g.,
\texttt{InCroMAP1.2.0.jar}. In this example, a minimum of 128\,MB and a maximum
of 1024\,MB of memory will be available for the program. In most cases,
InCroMAP needs more than 128\,MB memory, so it might be convenient to create a
shortcut and start the application with as much memory as available. If you
have 2\,GB RAM, for example, you might want to start the application with the
following command:

\begin{lstlisting}[language=bash,numbers=none]
java -Xms128m -Xmx1400M -jar InCroMAP.jar
\end{lstlisting}

For your convenience, we already created several start-scripts to run the
application with as much memory as possible. How much memory you actually need
strongly depends on the size of your input datasets.
%

\chapter{How to get started}

TODO: Write some text.

\section{General program features}
\subsection{Open a model}
\subsection{Starting the Application}
\subsection{Adjusting the Preferences}
\subsection{View the Results}
\subsection{Equation Generation One by One}


\chapter{Example use cases}

\section{\emph{De-novo} creation of kineitc equations}
\subsection{For an entire network}
\subsection{For selected reactions}
\section{Extraction of rate laws from SABIO-RK}
\subsection{For an entire network}
\subsection{For selected reations}

% 03_Troubleshooting
\chapter{FAQ / Troubleshooting}
\label{ch:faq}

TODO: These are some template questions. Add new ones and modify/ remove the
old ones to suit your needs.

\noindent \textbf{Where can I get help for a certain component/ option/ checkbox/ etc.?}\newline
Most elements in InCroMAP have tooltips. If you don't understand an option, you
can get help in the first place by just pointing the mouse cursor over it and
wait for the tooltip to show up ($\sim$ 3 seconds).\newline

\noindent \textbf{I'm getting a ``java.lang.OutOfMemoryError: Java heap space"}\newline
Some operations need a lot of memory. If you simply start InCroMAP, without any
JVM parameters, only 64\,MB of memory are available. Please append the argument
\texttt{-Xmx1024M} to start the application with 1\,GB of main memory. See
Section~\vref{startingTheProgram} for a more detailed description of how to
start the application with additional memory. If possible, you should give the
application 2\,GB of main memory. A minimum of 1\,GB main memory should be
available to the application.\newline

\noindent \textbf{Is an internet connection required to run InCroMAP?}\newline
An internet connection is required for most operations. Many identifier mapping
files and pathway-based visualizations require an active internet connection.
However, if you import your data directly with NCBI Entrez Gene IDs and do not
use the pathway-visualization or GO-enrichment, you should be able to run the
application offline.\newline

\noindent \textbf{Where can I get the latest version?}\newline
Go to \url{http://www.cogsys.cs.uni-tuebingen.de/software/Integrator/}.\newline

\noindent \textbf{Which Java version must be installed on my computer to launch
InCroMAP?}\newline InCroMAP requires at least Java 1.6. Please see
\url{http://www.java.com/de/download/} to download the latest Java version.

\noindent \textbf{Why does InCroMAP not start on my Mac with Mac OS prior to
10.6 Update 3?}\newline If you try to launch InCroMAP, but the application does
not start and you receive the following error message on the command-line or
Java console of your Mac, you need to update your Java installation:
\begin{verbatim}
Exception in thread "AWT-EventQueue-0" java.lang.NoClassDefFoundError:
    com/apple/eawt/AboutHandler
    at java.lang.ClassLoader.defineClass1(Native Method)
    at java.lang.ClassLoader.defineClass(ClassLoader.java:703)
    ...
\end{verbatim}
The interface \texttt{com.apple.eawt.AboutHandler} was introduced to Java for
Mac OS X 10.6 Update 3. If you have an earlier version of Mac OS or Java,
please update your OS or Java installation. Also see the Mac OS documentation
about the \texttt{AboutHandler} for more information. On a Mac, you can update
your Java installation through the Software Update menu item in the main Apple
menu.

\chapter{License}

SBMLsqueezer is free software: you can redistribute it and/or modify
it under the terms of the GNU General Public License as published by
the Free Software Foundation, either version~3 of the License, or
(at your option) any later version.

This program is distributed in the hope that it will be useful,
but \textbf{without any warranty}; without even the implied warranty of
\textbf{merchantability} or \textbf{fitness for a particular purpose}. See the
GNU General Public License for more details.

Each distributed version of SBMLsqueezer should contain a copy of the 
GNU General Public License. If not, please see
\href{http://www.gnu.org/licenses/gpl-3.0-standalone.html}{\nolinkurl{http://www.gnu.org/licenses/}}.

\chapter{Acknowledgments}

This work has been funded by the Federal Ministry of Education and Research
(BMBF, Germany) in the projects National Genome Research Network (NGFN-Plus,
project number 01GS08134) and the Virtual Liver Network (project number 0315756).

During the years, many people contributed to this project.
We are grateful to each contribution, such as source code, advice, proof reading
etc. Special thanks goes to Oliver Kohlbacher and Leif J. Pallesen.
The following list gives tribute to all people who have been working on the
implementation of this application.

\section{Core developers}

The following people implemented wide parts of SBMLsqueezer:
\begin{itemize}
\item Andreas Dr\"ager, 
  University of Tuebingen, Germany
  \href{mailto:andreas.draeger@uni-tuebingen.de}{andreas.draeger@uni-tuebingen.de}
\item Alexander D\"orr, 
  University of Tuebingen, Germany
  \href{mailto:alexander.doerr@uni-tuebingen.de}{alexander.doerr@uni-tuebingen.de}
\item Roland Keller,
  University of Tuebingen, Germany
  \href{mailto:roland.keller@uni-tuebingen.de}{roland.keller@uni-tuebingen.de}
\end{itemize}

\section{Contributors}

We like to acknowledge the valuable contribution of
\begin{itemize}
\item Johannes Eichner,
  University of Tuebingen, Germany
  \href{mailto:johannes.eichner@uni-tuebingen.de}{johannes.eichner@uni-tuebingen.de}
\item Andreas Zell, 
  University of Tuebingen, Germany
  \href{mailto:andreas.zell@uni-tuebingen.de}{andreas.zell@uni-tuebingen.de}
\end{itemize}

\section{Alumni}

We are greateful to our former colleagues and students
Meike Aichele,
Hannes Borch,
Nadine Hassis,
Marcel Kronfeld,
Sarah Rachel M\"uller vom Hagen,
Sebastian Nagel,
Alexander Peltzer,
Julianus Pfeuffer,
Matthias Rall,
Sandra Saliger,
Simon Sch\"afer,
Adrian Schr\"oder,
Jochen Supper,
Dieudonn\'e M. Wouamba,
Michael J. Ziller