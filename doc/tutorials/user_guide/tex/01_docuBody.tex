% 01_Introduction
\chapter{Overview}

The program SBMLsqueezer is a versatile tool for the assembly and assignment of
complex kinetic equations of reactions and processes in biochemical networks
\citep{Draeger2008, Draeger2010a, Draeger2011a}.
A list of currently implemented rate laws can be found in 
\vref{chap:RateLaws}.

The data format understood by SBMLsqueezer is the Systems Biology Markup
Language\footnote{\url{http://sbml.org}} (SBML) in all of its levels and
versions \citep{Hucka2001, Hucka2003, M.Hucka03012003, Hucka2007, Hucka2008,
Hucka2010a, Finney2003, Finney2006}.
All generated equations are stored, depending on the version of SBML, in form of
infix formula strings (SBML Level~1) or MathML \citep{Buswell1999} expressions
(for all other SBML versions).

Kinetic equations contain a set of parameters, whose units must be derived to
ensure that the overal equation can be interpreted in units of the extent of the
respective reaction per time units.
To give an example for the most common scenario, you can assume that the units
of the rate law will simplify to a substance unit per time unit, for instance,
mole per second.
Other constalations, however, are possible, too.
When using SBMLsqueezer you do not have to care much about units because the
program does this for you.
The program also annotates all created equations and parameters with appropriate
terms from the Systems Biology 
Ontology\footnote{\url{http://www.ebi.ac.uk/sbo/main/}} (SBO) and also
with MIRIAM-compliant\footnote{\url{http://co.mbine.org/standards/miriam}}
controlled vocabulary terms (Minimal Information Required In the Annotation of
Models), where appropriate \citep{Le2005, Novere2006b,
Laible2007, Courtot2011}.

The selection of the kinetic equations is done automatically by the program,
where several settings allow users to influence the algorithm's choice.
In many cases, SBMLsqueezer suggests multiple equations for the same process,
from which the user can choose as desired.
This behavior of the program is important to ensure that only appropriate
equations can be selected at any time.

In order to decide, which equations can be applied to a reaction of interest,
SBMLsqueezer analyzes several properties, such as the number and type of
reactants, products, and modifiers, if the reaction is reversible, etc.
Here, reaction of interest means that the user can either select individual
reactions from a larger model and equip these with kinetic equations, or rate
equations can be created in one go for the entire model.
Thereby, already existing equations can either be kept or overwritten, depending
on your choice.
You can influence and change all choices of the program, even in the one-go-mode.

If you are connected to the internet, you can also use SBMLsqueezer to extract
experimentally determined kinetic equations from the rate law database SABIO-RK
\citep[System for the Analysis of Biochemical Pathways -- Reaction Kinetics,][]{
Wittig2006, Rojas2007, Krebs2007, Wittig2012}.
To this end, SBMLsqueezer provides several settings, such as the organism,
temperature, or pH value under which the reaction's rate was determined. Again,
you can extract equations from SABIO-RK for an entire network in one go, or
individually select reactions of interest.

The program itself can be customized and used in several ways. It remembers your
last opened files, the window's size, your rate law selection and so forth.
You can run it as a stand-alone tool, or as a plug-in for the well-known program
CellDesigner\footnote{\url{http://www.celldesigner.org/}}
\citep{Funahashi2003, Funahashi2006, Funahashi2007a, Funahashi2008}.
By default, the stand-alone version is based on the \JSBML backend 
\citep{Draeger2011b}, but you can also use \libSBML \citep{Bornstein2008} as its
SBML backend instead if its more elaborated offline model validation matters.
If you neither like to download SBMLsqueezer, nor to install any software on
your local computer, you can even use SBMLsqueezer as a Galaxy-based webservice,
available at \url{http://webservices.cs.uni-tuebingen.de/}. 

Finally, the command-line mode of SBMLsqueezer provides all capabilities of the
program without any restriction.
Furhtermore, the fully accessible application programming interface can be used
to integrate SBMLsqueezer as an equation generating core into your end-user
program.
You can hence equip a large number of files with kinetic equations without the
need to open each file in a graphical user interface.
The usefulness of this approach has recently been demonstrated as part of the
path2models project\footnote{\url{http://www.ebi.ac.uk/biomodels-main/path2models}},
in which more than 142,000 SBML models have been processed with SBMLsqueezer.

For the documentation of your model, SBMLsqueezer includes the program
\SBMLLaTeX\footnote{\url{http://www.cogsys.cs.uni-tuebingen.de/software/SBML2LaTeX/}},
which generates a comprehensive report of your model, including a detailed
description of all components and equations \citep{Draeger2009b, Draeger2010a}.
These reports can be very handy to support scientific writing, because you can
easily copy the formulas into your paper.

\vspace{3cm}
\begin{center}
\includegraphics[width=2.5cm]{img/LOGO.png}
\end{center}

%%%%%%%%%%%%%%%%%%%%%%%%%%%%%%%%%%
% 02_Installation
%%%%%%%%%%%%%%%%%%%%%%%%%%%%%%%%%%

\chapter{Installation}

To obtain a local copy of SBMLsqueezer, you can download it in form of a 
\JavaArchive from the website
\url{http://www.cogsys.cs.uni-tuebingen.de/software/SBMLsqueezer/}.

In the most common scenario, you might want to launch the program as a
stand-alone tool and access its graphical user interface. To do so,
start SBMLsqueezer with a simple double click on the icon of the downloaded
\JAR.
Provided that a \JVM is installed on your system 
(see \vref{sec:SoftwareRequirements}), you will see the main window of
SBMLsqueezer as soon as the splash screen has finished.

\section{Requirements}

SBMLsqueezer can be be used in multiple ways, depending on your preferences:
\begin{itemize}
  \item As a stand-alone tool
  \begin{itemize}
    \item via its graphical user interface
    \item via its command-line interface
  \end{itemize}
        In both cases, you can choose between \JSBML or \libSBML as your backend
        for SBML.
  \item As a plug-in for CellDesigner
  \item As a gadget for GARUDA
  \item As an online program
  \item As a \JavaWebStart application
  \item As a rate law core in an end-user application
\end{itemize}
Depending on how you like to use SBMLsqueezer, differnt requirements must be
fulfilled before launching the program for the first time.

\subsection{Hardware}

With at least 1\,GB main memory, you should be able to perform most tasks
without any problem. For large models, you should have at least 2\,GB of main
memory. An active internet connection is required for accessing the SABIO-RK
database.

\subsection{Software}\label{sec:SoftwareRequirements}

SBMLsqueezer is entirely implemented in \Java and runs on any \OS, where a 
suitable \JVM, \JDK version~1.5 or newer, is installed.
For instructions how to obtain an up-to-date JVM for your system, see, for 
example, the \Java SE download
page\footnote{\url{http://www.oracle.com/technetwork/java/javase/downloads/}\label{fn:jvmldl}}.

SBMLsqueezer has successfully been tested with
\begin{itemize}
  \item Microsoft \WindowsSeven Professional (64~bit),
  \item Microsoft \WindowsSeven Professional (SP1, 64~bit),
  \item \MacOSX (version 10.8.2), and \UbuntuLinux (version 12.04, 64~bit).
\end{itemize}
See \vref{ch:faq} if you encounter any problems.

\subsection{Stand-alone application}

No special requirements are necessary if you like to use SBMLsqueezer as a
stand-alone application.
The program SBMLsqueezer does not have to be installed in order to be execuded, 
just copy the \JAR of SBMLsqueezer to your prefered path on your harddisk
to launch the application.
On \MacOSX, you may like to copy the \JAR into your \texttt{/Applications/}
folder.
On \Windows systems, the preferred position for the \JAR could be, for
instance, \texttt{C:\textbackslash Program\textbackslash{} Files\textbackslash}.
For \Linux, we propose to copy the \JAR to the \texttt{/opt/} folder.
 

\subsubsection{Using the \JSBML backend}

No special actions are necessary if you like to use \JSBML, because this is the
default and \JSBML is already included in the \JAR that you have downloaded.

\subsubsection{Launching SBMLsqueezer with \libSBML as backend}

Follow the installation instructions of the \Java binding for \libSBML for your
platform, which you can find at the website of
\libSBML\footnote{\url{http://sbml.org/Software/libSBML}}.
On some platforms, you may have to define an environment variable pointing
to the installation directory of \libSBML before being able to use its \Java
binding.
On the most \Unix and \Linux platforms, this variable is called
\texttt{LD\_LIBRARY\_PATH}.
On \MacOSX, you should instead define the variable \texttt{DYLD\_LIBRARY\_PATH}.
Follow the instrcutions at \url{http://sbml.org/Software/libSBML/docs/cpp-api/libsbml-accessing.html#accessing-java}
before launching SBMLsqueezer.
On Windows you should add the directory to the PATH environment in the Control Panel.
The following skript can help you to to run SBMLsqueezer on your \Unix platform:
\begin{lstlisting}[language=bash]
VM_ARGS="-Xms32M -Xmx512M -Djava.library.path="

# The following lines depend on your system's configuration; 
# so here is just an example:
VM_ARGS="${VM_ARGS}\
/usr/lib/jvm/java-6-sun/jre/lib/i386/client:\
/usr/lib/jvm/java-6-sun/jre/lib/i386:\
/usr/lib/jvm/java-6-sun/lib:\
/usr/local/lib"
#:[path to xerces]/xerces/lib
CLASS_PATH="/usr/local/share/java/libsbmlj.jar:\
[path to ]SBMLsqueezer2.jar"

# Set the environment variable; under Linux or most Unix systems this is
LD_LIBRARY_PATH="${LD_LIBRARY_PATH}:/usr/local/lib"
# On Mac OS you have to use the following code instead:
#DYLD_LIBRARY_PATH="${DYLD_LIBRARY_PATH}:/usr/local/lib"

# Start SBMLsqueezer using the command-line options:
MAIN_CLASS=org.sbml.squeezer.SBMLsqueezer 
java ${VM_ARGS} -cp ${CLASS_PATH} ${MAIN_CLASS} --try-loading-libsbml=true\
[further options]
\end{lstlisting}
In the example above the arguments for the \JVM define an initial heap space of
32~MB (\texttt{-Xms32M}) and a maximal heap size of 512~MB (\texttt{-Xmx512M}).

\subsection{Plug-in for CellDesigner}



\subsection{Integration into GARUDA}

\section{Starting the application}
\label{startingTheProgram}

If you downloaded a ZIP-file, you need to unzip it before starting the
application. Depending on your \OS, you should use the provided
shell scripts for starting the application. This is \texttt{start.sh} for \Linux
or \texttt{start.bat} for \Windows. On \MacOSX, you have to create your own
shortcut. You can start the application on all {\OS}s by typing

\begin{lstlisting}[language=bash,numbers=none]
java -jar -Xms128m -Xmx1024m SBMLsqueezer.jar
\end{lstlisting}

\noindent on your command prompt. Please note that you might have to change
\texttt{SBMLsqueezer.jar} for the real name of the \JAR, e.g.,
\texttt{SBMLsqueezer2.0.jar}. In this example, a minimum of 128\,MB and a maximum
of 1024\,MB of memory will be available for the program. In most cases,
SBMLsqueezer needs more than 128\,MB memory, so it might be convenient to create a
shortcut and start the application with as much memory as available. If you
have 2\,GB RAM, for example, you might want to start the application with the
following command:

\begin{lstlisting}[language=bash,numbers=none]
java -Xms128m -Xmx1400M -jar SBMLsqueezer.jar
\end{lstlisting}

For your convenience, we already created several start-scripts to run the
application with as much memory as possible. How much memory you actually need
strongly depends on the size of your input datasets.
%

\chapter{How to get started}

TODO: Write some text.

\section{General program features}
\subsection{Open a model}
\subsection{Starting the Application}
\subsection{Adjusting the Preferences}
\subsection{View the Results}
\subsection{Equation Generation One by One}


\chapter{Example use cases}

\section{\emph{De-novo} creation of kinetic equations}
\subsection{For an entire network}
\subsection{For selected reactions}
\section{Extraction of rate laws from SABIO-RK (does not apply for CellDesigner plugin)}
\subsection{For an entire network}
Please note: The reactions you want to equip with kinetic equations should be annotated with KEGG reaction ids.
Click on the SABIO-RK symbol and the wizard will be started. On the first window you can select the reactions you want to equip with kinetic equations from SABIO-RK. Press Ctrl while clicking on these reactions and click on NEXT when finished. The KEGG id of a reaction is always included in the respective search, as the API searches for rate equations for several reactions simultaneously. On the next window you can add search terms that should apply for the rate equations of all reactions such as the organism by choosing from the provided terms and typing in the value of the term. You can remove a query field again by clicking on DELETE. Your search can be further restricted by changing the bounds of pH, temperature, etc. in the right panel. On the following window you see for which reactions matching entries have been found. A reaction is marked in green, if a matching rate equation has been found, yellow, if the found rate equations do not fit exactly to your model, and red, if no matching rate equation has been found. If you are satisfied with the results, you can click on NEXT. A summary of all necessary changes to the model for adding rate equations to the reactions marked in green, is displayed. If you now click on FINISH and confirm the changes afterwards, the rate equations will be added and the necessary changes will be applied to your model.

\subsection{For selected reations}
Select a reaction and right-click on it. Then click on the SABIO-RK symbol and the wizard will be started. The reaction is shown again in the first window and you can just click on NEXT. A window for selecting search terms is shown afterwards. You can add search tearms such as the organism by choosing from the provided terms and typing in the value of the term. The search can be further restricted by changing the bounds of pH, temperature, etc. in the right panel. You can remove a query field by clicking on DELETE. After you are satisfied with your choice of search terms, click on NEXT. The rate equations that match your search criteria are displayed then. You can select a reaction and click on NEXT or you can click on BACK and change your search criteria. If you click on NEXT, you can then choose how to match the species, compartments, etc., contained in the selected rate equation to the respective SBML elements in your model. The necessary elements to import, such as function definitions contained in the rate equation to add, are also shown. If you now click on NEXT, a summary of all necessary changes to the model for adding the selected rate equation is displayed. Now click on FINISH  and confirm the changes afterwards. Then the rate equation will be added and the necessary changes will be applied to your model.

\chapter{Command-line arguments}

\chapter{Supported rate laws}\label{chap:RateLaws}

The kinds of equations supported by the program are
numerous, including traditional approaches \citep{Guldberg1879, Michaelis1913}
just like very recent equations \citep{Liebermeister2006, Liebermeister2010}.
It provides equations for gene-regulatory processes
\citep{Hinze2007, Radde2007a, Toepfer2007, Vu2007,Weaver1999} and approximative
rate laws \citep{Savageau1969}.
In addition, SBMLsqueezer covers a large variety of standard rate laws for
biochemical reactions from relevant text books
\citep{Segel1993, Heinrich1996, Bisswanger2000, Cornish-Bowden2004}.
Here, we give a short overview of all equations that are currently implemented
in SBMLsqueezer, ordered by the categories metabolic or gene-regulatory.
However, the actual algorithm that suggests applicable rate laws is more complex
and considers several features of the reaction.
It can therefore happen that SBMLsqueezer suggests multiple different rate
equations for the same reaction.

\section{Rate laws for metabolic processes}
\begin{itemize}
  \item (Generalized) mass-action rate law with numerous orders \citep[p.~16]{Guldberg1879, Heinrich1996}
  \item Uni-uni Michaelis-Menten kinetics \citep{Michaelis1913}
  \item Irreversible non-modulated non-interacting reactant enzymes (SBO)
  \item Bi-uni enzyme mechanisms \citep{Segel1993, Bisswanger2000, Cornish-Bowden2004}
  \begin{itemize}
    \item Random-order mechanism
    \item Ordered mechanism
  \end{itemize}
  \item Bi-bi enzyme reactions \citep{Segel1993, Bisswanger2000, Cornish-Bowden2004}
  \begin{itemize}
    \item Random-order mechanism \citep[p.~169]{Cornish-Bowden2004}
    \item Ordered mechanism
    \item Ping-pong mechanism
  \end{itemize}
  \item Modular rate laws for enzymati reactions \citep{Liebermeister2010}
  \begin{itemize}
    \item Power-law modular rate law (PM)
    \item Common modular rate law (CM)
    \item Direct binding modular rate law (DM)
    \item Simultaneous binding modular rate law (SM)
    \item Force-dependent modular rate law (FM)
  \end{itemize}
  \item Convenience kinetics \citep{Liebermeister2006}
  \begin{itemize}
    \item Thermodynamically dependent form
    \item Thermodynamically independent form
  \end{itemize}
  \item (Generalized) Hill equation \citep[p.~314]{Hill1910, Cornish-Bowden2004}
\end{itemize}

\section{Rate laws for gene-regulatory processes}
\begin{itemize}
  \item Hill-Hinze equation \citep{Hinze2007}
  \item Hill-Radde equation \citep{Radde2007a, Radde2007}
  \item Linear additive network models
    \begin{itemize}
      \item General form
      \item NetGenerator form \citep{Toepfer2007}
    \end{itemize}
  \item Non-linear additive network models
    \begin{itemize}
      \item General form
      \item NetGenerator form \citep{Toepfer2007}
      \item Vohradsk{\'y}'s equation \citep{Vu2007}
      \item Weaver's equation \citep{Weaver1999}
    \end{itemize}
  \item S-systems \citep{Savageau1969, spieth04optimizing, Tournier2005, Spieth2006, Hecker2009}
  \item H-systems \citep{Spieth2006}
\end{itemize}


% 03_Troubleshooting
\chapter{FAQ and troubleshooting}
\label{ch:faq}

TODO: These are some template questions. Add new ones and modify/ remove the
old ones to suit your needs.

\noindent \textbf{Where can I get help for a certain component/ option/ checkbox/ etc.?}\newline
Most elements in SBMLsqueezer have tooltips. If you don't understand an option, you
can get help in the first place by just pointing the mouse cursor over it and
wait for the tooltip to show up ($\sim$ 3 seconds).\newline

\noindent \textbf{I'm getting a ``java.lang.OutOfMemoryError: Java heap space"}\newline
Some operations need a lot of memory. If you simply start SBMLsqueezer, without any
JVM parameters, only 64\,MB of memory are available. Please append the argument
\texttt{-Xmx1024M} to start the application with 1\,GB of main memory. See
\vref{startingTheProgram} for a more detailed description of how to
start the application with additional memory. If possible, you should give the
application 2\,GB of main memory. A minimum of 1\,GB main memory should be
available to the application.\newline

\noindent \textbf{Is an internet connection required to run SBMLsqueezer?}\newline
An internet connection is required for most operations. Many identifier mapping
files and pathway-based visualizations require an active internet connection.
However, if you import your data directly with NCBI Entrez Gene IDs and do not
use the pathway-visualization or GO-enrichment, you should be able to run the
application offline.\newline

\noindent \textbf{Where can I get the latest version?}\newline
Go to \url{http://www.cogsys.cs.uni-tuebingen.de/software/Integrator/}.\newline

\noindent \textbf{Which \Java version must be installed on my computer to launch
SBMLsqueezer?}\newline SBMLsqueezer requires at least \Java 1.6. Please see
\url{http://www.java.com/de/download/} to download the latest \Java version.

\noindent 
\textbf{Why does SBMLsqueezer not start on my Mac with \MacOSX prior to 10.6 Update 3?}\newline If you try to launch SBMLsqueezer, but the application does
not start and you receive the following error message on the command-line or
\Java console of your Mac, you need to update your \Java installation:
\begin{verbatim}
Exception in thread "AWT-EventQueue-0" java.lang.NoClassDefFoundError:
    com/apple/eawt/AboutHandler
    at java.lang.ClassLoader.defineClass1(Native Method)
    at java.lang.ClassLoader.defineClass(ClassLoader.java:703)
    ...
\end{verbatim}
The interface \texttt{com.apple.eawt.AboutHandler} was introduced to \Java for
\MacOSX 10.6 Update 3. If you have an earlier version of \MacOSX or \Java,
please update your OS or \Java installation. Also see the \MacOSX documentation
about the \texttt{AboutHandler} for more information. On a Mac, you can update
your \Java installation through the Software Update menu item in the main Apple
menu.

\chapter{License}

SBMLsqueezer is free software: you can redistribute it and/or modify
it under the terms of the GNU General Public License as published by
the Free Software Foundation, either version~3 of the License, or
(at your option) any later version.

This program is distributed in the hope that it will be useful,
but \textbf{without any warranty}; without even the implied warranty of
\textbf{merchantability} or \textbf{fitness for a particular purpose}. See the
GNU General Public License for more details.

Each distributed version of SBMLsqueezer should contain a copy of the 
GNU General Public License. If not, please see
\href{http://www.gnu.org/licenses/gpl-3.0-standalone.html}{\nolinkurl{http://www.gnu.org/licenses/}}.

\chapter{Acknowledgments}

This work has been funded by the Federal Ministry of Education and Research
(BMBF, Germany) in the projects National Genome Research Network (NGFN-Plus,
project number 01GS08134) and the Virtual Liver Network (project number 0315756).

During the years, many people contributed to this project.
We are grateful to each contribution, such as source code, advice, proof reading
etc. Special thanks goes to Oliver Kohlbacher and Leif J. Pallesen.
The following list gives tribute to all people who have been working on the
implementation of this application.

\section{Core developers}

The following people implemented wide parts of SBMLsqueezer:
\begin{itemize}
\item Andreas Dr\"ager, 
  University of Tuebingen, Germany
  \href{mailto:andreas.draeger@uni-tuebingen.de}{andreas.draeger@uni-tuebingen.de}
\item Alexander D\"orr, 
  University of Tuebingen, Germany
  \href{mailto:alexander.doerr@uni-tuebingen.de}{alexander.doerr@uni-tuebingen.de}
\item Roland Keller,
  University of Tuebingen, Germany
  \href{mailto:roland.keller@uni-tuebingen.de}{roland.keller@uni-tuebingen.de}
\end{itemize}

\section{Contributors}

We like to acknowledge the valuable contribution of
\begin{itemize}
\item Johannes Eichner,
  University of Tuebingen, Germany
  \href{mailto:johannes.eichner@uni-tuebingen.de}{johannes.eichner@uni-tuebingen.de}
\item Andreas Zell, 
  University of Tuebingen, Germany
  \href{mailto:andreas.zell@uni-tuebingen.de}{andreas.zell@uni-tuebingen.de}
\end{itemize}

\section{Alumni}

We are greateful to our former colleagues and students
Meike Aichele,
Hannes Borch,
Nadine Hassis,
Marcel Kronfeld,
Sarah Rachel M\"uller vom Hagen,
Sebastian Nagel,
Alexander Peltzer,
Julianus Pfeuffer,
Matthias Rall,
Sandra Saliger,
Simon Sch\"afer,
Adrian Schr\"oder,
Jochen Supper,
Dieudonn\'e M. Wouamba,
Michael J. Ziller