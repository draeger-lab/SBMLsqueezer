% 01_Introduction
\chapter{Overview}

The program SBMLsqueezer is a versatile tool for the assembly and assignment of
complex kinetic equations of reactions and processes in biochemical networks
\citep{Draeger2008, Draeger2010a, Draeger2011a}.
A list of currently implemented rate laws can be found in \vref{chap:RateLaws}.

The data format understood by SBMLsqueezer is the
\SBML\footnote{\url{http://sbml.org}} in all of its levels and
versions \citep{Hucka2001, Hucka2003, M.Hucka03012003, Hucka2007, Hucka2008,
Hucka2010a, Finney2003, Finney2006}.
When using \SBML Level~1 (not recommended), all generated equations are stored
in form of infix formula strings. For all other versions of \SBML, the generated
equations are stored in form of MathML \citep{Buswell1999} expressions.

Kinetic equations contain a set of parameters, whose units must be derived to
ensure that the overal equation can be interpreted in units of the extent of the
respective reaction per time units.
To give an example for the most common scenario, you can assume that the units
of the rate law will simplify to a substance unit per time unit, for instance,
mole per second.
Other constalations, however, are possible, too.
When using SBMLsqueezer you do not have to care much about units because the
program does this for you.
The program also annotates all created equations and parameters with appropriate
terms from the \SBO\footnote{\url{http://www.ebi.ac.uk/sbo/main/}} and also
with \MIRIAM-compliant\footnote{\url{http://co.mbine.org/standards/miriam}}
controlled vocabulary terms (Minimal Information Required In the Annotation of
Models), where appropriate \citep{Le2005, Novere2006b,
Laible2007, Courtot2011}.

The selection of the kinetic equations is done automatically by the program,
where several settings allow users to influence the algorithm's choice.
In many cases, SBMLsqueezer suggests multiple equations for the same process,
from which the user can choose as desired one.
This behavior of the program is important to ensure that only appropriate
equations can be selected at any time.

In order to decide, which equations can be applied to a reaction of interest,
SBMLsqueezer analyzes several properties, such as the number and type of
reactants, products, and modifiers, if the reaction is reversible, etc.
Here, reaction of interest means that the user can either select individual
reactions from a larger model and equip these with kinetic equations, or rate
equations can be created in one go for the entire model.
Thereby, already existing equations can either be kept or overwritten, depending
on your choice.
You can influence and change all choices of the program, even in the one-go-mode.

If you are connected to the internet, you can also use SBMLsqueezer to extract
experimentally determined kinetic equations from the rate law database \SABIO
\citep{Wittig2006, Rojas2007, Krebs2007, Wittig2012}.
To this end, SBMLsqueezer provides several settings, such as the organism,
temperature, or pH value under which the reaction's rate was determined. Again,
you can extract equations from \SABIO for an entire network in one go, or
individually select reactions of interest.

The program itself can be customized and used in several ways. It remembers your
last opened files, the window's size, your rate law selection and so forth.
You can run it as a stand-alone tool, or as a plug-in for the well-known program
\CellDesigner\footnote{\url{http://www.celldesigner.org/}}
\citep{Funahashi2003, Funahashi2006, Funahashi2007a, Funahashi2008}.
By default, the stand-alone version is based on the \JSBML backend 
\citep{Draeger2011b}, but you can also use \libSBML \citep{Bornstein2008} as its
\SBML backend instead if this library's more elaborated offline model validation
matters.
If you neither like to download SBMLsqueezer, nor to install any software on
your local computer, you can even use SBMLsqueezer as a Galaxy-based
webservice\footnote{Galaxy webservice available at \url{http://webservices.cs.uni-tuebingen.de/}},
or launch it as a \JavaWebStart application directly from your web
browser\footnote{You can use SBMLsqueezer as a \JavaWebStart application by
clicking at \url{http://www.cogsys.cs.uni-tuebingen.de/software/SBMLsqueezer/downloads/SBMLsqueezer.jnlp}.}.

In addition, SBMLsqueezer implements the \Garuda specification and can therefore
also be used as a gadget within this powerful framework for systems biology
\citep{Ghosh2011}. In this configuration, the output of tools that assist you
to create your models, such as KEGGtranslator \citep{Wrzodek2011, Wrzodek2013}
can directly be piped as the input to SBMLsqueezer. Furthermore, SBMLsqueezer's
output can be forwarded to further programs, such as SBMLsimulator
\citep{Keller2013} for further analysis.

Finally, the command-line mode of SBMLsqueezer provides all capabilities of the
program without any restriction.
Furhtermore, the fully accessible application programming interface can be used
to integrate SBMLsqueezer as an equation generating core into your end-user
program.
You can hence equip a large number of files with kinetic equations without the
need to open each file in a graphical user interface.
The usefulness of this approach has recently been demonstrated as part of the
path2models project\footnote{\url{http://www.ebi.ac.uk/biomodels-main/path2models}},
in which more than 142,000 \SBML models have been processed with SBMLsqueezer
\citep{Buechel2013}.

For the documentation of your model, SBMLsqueezer includes the program
\SBMLLaTeX\footnote{\url{http://www.cogsys.cs.uni-tuebingen.de/software/SBML2LaTeX/}},
which generates a comprehensive report of your model, including a detailed
description of all components and equations \citep{Draeger2009b, Draeger2010a}.
These reports can be very handy to support scientific writing, because you can
easily copy the formulas into your scientific paper.

This document is intended to guide you through installation and use of
SBMLsqueeer by introducing its various interfaces to you. The next chapter
describes how you can install and access SBMLsqueezer on your \OS.
\vref{chap:GUI} introduces the graphical user interface of the stand-alone
version of SBMLsqueezer to you and explains how to access all of its functions.
If you are interested to run SBMLsqueezer in a batch or shell script, the
command-line interface will be interesting for you, which is described in
\vref{chap:CMD}.

Just as every other scientific software, also SBMLsqueezer is work in progress.
If you encounter any difficulties or bugs, please do not hesitate to contact
the mailing list,
\href{mailto:sbmlsqueezer@googlegroups.com}{\texttt{sbmlsqueezer@google\-groups.com}}. 


% \vspace{3cm}
% \begin{center}
% \includegraphics[width=2.5cm]{img/LOGO.png}
% \end{center}

%%%%%%%%%%%%%%%%%%%%%%%%%%%%%%%%%%
% 02_Installation
%%%%%%%%%%%%%%%%%%%%%%%%%%%%%%%%%%

\chapter{Installation}

To obtain a local copy of SBMLsqueezer, you can download it in form of a 
\JAR from the website
\url{http://www.cogsys.cs.uni-tuebingen.de/software/SBMLsqueezer/}.

In the most common scenario, you might want to launch the program as a
stand-alone tool and access its graphical user interface. To do so,
start SBMLsqueezer with a simple double click on the icon of the downloaded
\JAR.
Provided that a \JVM is installed on your system 
(see \vref{sec:SoftwareRequirements}), you will see the main window of
SBMLsqueezer as soon as the splash screen has finished.

\section{Requirements}

SBMLsqueezer can be be used in multiple ways, depending on your preferences:
\begin{itemize}
  \item As a stand-alone tool
  \begin{itemize}
    \item via its graphical user interface (see \vref{chap:GUI})
    \item via its command-line interface (see \vref{chap:CMD})
  \end{itemize}
        In both cases, you can choose between \JSBML or \libSBML as your backend
        for \SBML (see \vref{sec:StandAlone}).
  \item As a plug-in for \CellDesigner (see \vref{sec:CellDesignerInstall})
  \item As a gadget for \Garuda (see \vref{sec:GarudaInstall})
  \item As an online program embedded in the Galaxy webservice environment
  \item As a \JavaWebStart application
  \item As a rate law core in an end-user application via its \API
\end{itemize}
Depending on how you like to use SBMLsqueezer, differnt requirements must be
fulfilled before launching the program for the first time.

\subsection{Hardware}

With at least 1\,GB main memory, you should be able to perform most tasks
without any problem. For large models, you should have at least 2\,GB of main
memory. An active internet connection is required for accessing the \SABIO
database and to automatically obtain update information for SBMLsqueezer.

\subsection{Software}\label{sec:SoftwareRequirements}

SBMLsqueezer is entirely implemented in \Java and runs on any \OS, where a 
suitable \JVM, \JDK version~1.6 or newer, is installed.
For instructions how to obtain an up-to-date \JVM for your system, see, for 
example, the \Java SE download
page\footnote{\url{http://www.oracle.com/technetwork/java/javase/downloads/}\label{fn:jvmldl}}.

SBMLsqueezer has successfully been tested with
\begin{itemize}
  \item Microsoft \WindowsSeven Professional (64~bit),
  \item Microsoft \WindowsSeven Professional (SP1, 64~bit),
  \item \MacOSX (versions 10.8.2 through 10.9.2), and \UbuntuLinux (version 12.04, 64~bit).
\end{itemize}
See \vref{ch:faq} if you encounter any problems.

\subsection{Stand-alone application}
\label{sec:StandAlone}

Besides what is explained above, no further requirements are necessary if you
like to use SBMLsqueezer as a stand-alone application.
The program SBMLsqueezer does not have to be installed in order to be execuded, 
just copy the \JAR of SBMLsqueezer to your prefered path on your harddisk
to launch the application.
On \MacOSX, you may like to copy the \JAR into your \texttt{/Applications/}
folder.
On \Windows systems, the preferred position for the \JAR could be, for
instance, \texttt{C:\textbackslash Program{\textvisiblespace}Files\textbackslash}.
For \Linux, we propose to copy the \JAR to the \texttt{/opt/} folder.


\subsubsection{Using the \JSBML backend}

No special actions are necessary if you like to use \JSBML, because this is the
default and \JSBML is already included in the \JAR that you have downloaded.

\subsubsection{Launching SBMLsqueezer with \libSBML as backend}

Follow the installation instructions of the \Java binding for \libSBML for your
platform, which you can find at the website of
\libSBML\footnote{\url{http://sbml.org/Software/libSBML}}.
On some platforms, you may have to define an environment variable pointing
to the installation directory of \libSBML before being able to use its \Java
binding.
On the most \Unix and \Linux platforms, this variable is called
\LDLIBRARYPATH.
On \MacOSX, you should instead define the variable \DYLDLIBRARYPATH.
Follow the instrcutions at \url{http://sbml.org/Software/libSBML/docs/cpp-api/libsbml-accessing.html#accessing-java}
before launching SBMLsqueezer.
On Windows you should add the directory to the \PATH environment in the Control Panel.
The following script can help you to to run SBMLsqueezer on your \Unix platform:
\begin{lstlisting}[language=bash]
VM_ARGS="-Xms32M -Xmx512M -Djava.library.path="

# The following lines depend on your system's configuration; 
# so here is just an example:
VM_ARGS="${VM_ARGS}\
/usr/lib/jvm/java-6-sun/jre/lib/i386/client:\
/usr/lib/jvm/java-6-sun/jre/lib/i386:\
/usr/lib/jvm/java-6-sun/lib:\
/usr/local/lib"
#:[path to xerces]/xerces/lib
CLASS_PATH="/usr/local/share/java/libsbmlj.jar:\
[path to ]SBMLsqueezer_v2.0.jar"

# Set the environment variable; under Linux or most Unix systems this is
LD_LIBRARY_PATH="${LD_LIBRARY_PATH}:/usr/local/lib"
# On Mac OS you have to use the following code instead:
#DYLD_LIBRARY_PATH="${DYLD_LIBRARY_PATH}:/usr/local/lib"

# Start SBMLsqueezer using the command-line options:
MAIN_CLASS=org.sbml.squeezer.SBMLsqueezer 
java ${VM_ARGS} -cp ${CLASS_PATH} ${MAIN_CLASS} --try-loading-libsbml=true\
[further options]
\end{lstlisting}
In the example above the arguments for the \JVM define an initial heap space of
32~MB (\texttt{-Xms32M}) and a maximal heap size of 512~MB (\texttt{-Xmx512M}).

\subsection{Plug-in for \CellDesigner}
\label{sec:CellDesignerInstall}

If you like to use SBMLsqueezer as a plugin for the popular graphical model
editor \CellDesigner, you can do this by following these installation steps.
\begin{enumerate}
  \item Download and install \CellDesigner from \url{http://celldesigner.org}.
  \item Depending on your \OS, \CellDesigner will be installed in different
        folders. Locate, where your copy of \CellDesigner is installed. In
        \Windows it is usually \texttt{C:\textbackslash Program{\textvisiblespace}Files\textbackslash CellDesigner4.3}, in
        \Linux you might find it in \texttt{/opt/CellDesigner4.3}, and in
        \MacOSX \CellDesigner will be installed under
        \texttt{/Applications/CellDesigner4.3}, where 4.3 is the version number
        with which SBMLsqueezer 2.0 has been tested.
  \item Open the \texttt{plugin} folder in your installation of \CellDesigner and
        copy the file \texttt{SBMLsqueezer\_v2.0\_incl-libs.jar} into this folder.
  \item Open the \texttt{lib} folder in your installation of \CellDesigner and
        replace the \JSBML \JAR that is redistributed by \CellDesigner with
        the \JAR \href{http://www.cogsys.cs.uni-tuebingen.de/software/SBMLsqueezer/downloads/jsbml-1.0-a1-with-dependencies.jar}{jsbml-1.0-a1-with-dependencies.jar}
        that you can download at \url{http://www.cogsys.cs.uni-tuebingen.de/software/SBMLsqueezer/downloads/}.
        Make sure that the name of the new \JSBML file will be identical to the
        name of the \JSBML file distributed with \CellDesigner. Otherwise, you
        will have to change some start script of \CellDesigner. Even though this
        is not difficult and can be done on each \OS, we prefer to avoid that
        for the sake of simplicity. Alternatively, you can also download \JSBML
        from the official sourceforge page. However, you must make sure that the
        downloaded \JAR includes the package \texttt{org.sbml.jsbml.celldesigner}.
        Without this package, SBMLsqueezer cannot be launched in \CellDesigner.
\end{enumerate}


\subsection{Integration into \Garuda}
\label{sec:GarudaInstall}

When downloading \Garuda, a functioning copy of SBMLsqueezer will be included.
Hence, no special installation steps are required. Just follow the instructions
of how to install \Garuda.

\section{Starting the application}
\label{startingTheProgram}

If you downloaded a ZIP-file, you need to unzip it before starting the
application. In the simplest case, you can launch SBMLsqueezer just by double
clicking on its application icon. This will launch the graphical user interface
(see \vref{chap:GUI} for further details).

Depending on your preference and your \OS, it might be helpful for
you to write a short shell or bash scripts for starting the application.
You might name this, for instance, \texttt{start.sh} for \Linux or \MacOSX, and
\texttt{start.bat} for \Windows. Within those scripts you could specify several
command-line options (see \vref{chap:CMD}) and hence customize the behavior of
SBMLsqueezer. 

You can start the application on all {\OS}s by typing
\begin{lstlisting}[language=bash,numbers=none]
java -jar -Xms128m -Xmx1024m SBMLsqueezer.jar
\end{lstlisting}
on your command prompt. Please note that you might have to change
\texttt{SBMLsqueezer.jar} for the real name of the \JAR, e.g.,
\texttt{SBMLsqueezer\_v2.0\_incl-libs.jar}. In this example, a minimum of 128\,MB
and a maximum of 1024\,MB of memory will be available for the program. In most cases,
SBMLsqueezer needs more than 128\,MB memory, so it might be convenient to create a
shortcut and start the application with as much memory as available. If you
have 2\,GB RAM, for example, you might want to start the application with the
following command:
\begin{lstlisting}[language=bash,numbers=none]
java -Xms128m -Xmx1400M -jar SBMLsqueezer.jar
\end{lstlisting}
How much memory you actually need strongly depends on the size of your input datasets.
%

\chapter{How to get started}
\label{chap:GUI}

TODO: Write some text.

\section{General program features}
\subsection{Open a model}
\subsection{Starting the application}
\subsection{Adjusting the preferences}
\subsection{View the results}
\subsection{Equation generation one by one}
\begin{figure}[htbp]
\shadowimage[width=\textwidth]{Screenshot_2}
\caption{Generating kinetic a equation for a single reaction. This dialog can
be opened by right-clicking on an individual reaction in your active model.
Here SBMLsqueezer analyzes the current reaction according to various properties
and displays a selection of suitable rate laws that, of which you can select
the most appropriate one. The equation preview, where you can zoom in and out
of the rendered formula, helps you to choose a rate law. If a corresponding
\SBO entry can be found for a rate law, a tooltip will be displayed with further
information about this equation.
You can also decide if the reversibility of the reaction should be changed.
This would also alter the list of applicable kinetic equations.
The other options, if newly created
parameters should be locally attached to the kinetic law object or globally to
the model influences the internal structure of the model, but not its behavior.
Just note, that there are some parameters, which cannot be stored locally because
they represent recurrent properties of some reactive species and are needed
accross several rate laws (e.g., energy constants).}
\label{fig:RateLawDialog}
\end{figure}

\subsection{Generate kinetic equations in a single step}
\begin{figure}[htbp]
\shadowimage[width=\textwidth]{Screenshot_4}
\caption{Generating kinetic equations in one single step with the kinetic law wizard.
This screenshot shows the Kinetic Law Wizard (here shown with the user interface
under \MacOSX). When launched either via the tool bar by clicking at the lemon
icon or the menu bar in the menu (Edit | Squeeze), this wizard allows you to
generate kinetic equations for all reactions in the active model. Various options
allow you to customize how this is done. The most important options can be used
to decide if already existing rate laws should either be kept or overwritten,
if all reactions can be modeled in a reversible maner or as currently
defined in the model.}
\label{fig:KineticLawWizard}
\end{figure}


\chapter{Example use cases}

\section{\emph{De-novo} creation of kinetic equations}
\subsection{For an entire network}
\subsection{For selected reactions}
\section{Extraction of rate laws from \SABIO}

Note that this feature does not apply for \CellDesigner plugin, because \CellDesigner brings already with it its own \SABIO interface.

\subsection{For an entire network}
Please note: The reactions you want to equip with kinetic equations should be annotated with KEGG reaction ids.

\begin{figure}[htbp]
\shadowimage[width=\textwidth]{SABIO_automatic_1}
\caption{Starting the SABIO application. You have to click on the \SABIO symbol and the wizard will be started.}
\label{fig:startAutomatic}
\end{figure}

\begin{figure}[htbp]
\shadowimage[width=\textwidth]{SABIO_automatic_2}
\caption{Selection of reation.On the first window you can select the reactions you want to equip with kinetic equations from \SABIO. 
Press Ctrl while clicking on these reactions and click on NEXT when finished.}
\label{fig:selectReations}
\end{figure}

\begin{figure}[htbp]
\shadowimage[width=\textwidth]{SABIO_automatic_5}
\caption{Selection of search terms. On the next window you can add search terms that should apply for the rate equations of all reactions 
such as the organism by choosing from the provided terms and typing in the value of the term. The KEGG id of a reaction is always included 
in the respective search, as the API searches for rate equations for several reactions simultaneously. Your search can be further restricted
by changing the bounds of pH, temperature, etc. in the right panel.}
\label{fig:searchTerms}
\end{figure}

\begin{figure}[htbp]
\shadowimage[width=\textwidth]{SABIO_automatic_6}
\caption{Removal of search terms. You can remove a query field again by clicking on DELETE.}
\label{fig:removeSearchTerms}
\end{figure}

\begin{figure}[htbp]
\shadowimage[width=\textwidth]{SABIO_automatic_8}
\caption{Window with found kinetic equations. On the following window you see for which reactions matching entries have been found.
A reaction is marked in green, if a matching rate equation has been found, yellow, if the found rate equations do not fit exactly to your 
model, and red, if no matching rate equation has been found. If you are satisfied with the results, you can click on NEXT.}
\label{fig:foundEquations}
\end{figure}


\begin{figure}[htbp]
\shadowimage[width=\textwidth]{SABIO_automatic_9}
\caption{Summary of necessary changes. A summary of all necessary changes to the model for adding rate equations to the reactions 
marked in green, is displayed.If you now click on FINISH and confirm the changes afterwards, the rate equations will be added and 
the necessary changes will be applied to your model.}
\label{fig:changes}
\end{figure}




\subsection{For selected reations}
\begin{figure}[htbp]
\shadowimage[width=\textwidth]{SABIO_manual_1}
\caption{Starting the SABIO application for a selected reaction. Select a reaction and right-click on it. Then click on the \SABIO symbol 
and the wizard will be started. The reaction is shown again in the first window and you can just click on NEXT.}
\label{fig:startManual}
\end{figure}

\begin{figure}[htbp]
\shadowimage[width=\textwidth]{SABIO_manual_3}
\caption{Selection of search terms. You can add search tearms such as the organism by choosing from the provided terms and typing in 
the value of the term. The search can be further restricted by changing the bounds of pH, temperature, etc. in the right panel. 
You can remove a query field by clicking on DELETE (like when adding kinetic laws to several reactions simultaneously).}
\label{fig:searchTermsManual}
\end{figure}

\begin{figure}[htbp]
\shadowimage[width=\textwidth]{SABIO_manual_7}
\caption{Window with two search terms. Here a search for an organism and an enzymename has been specified and will be combined to a query when clicking on NEXT.}
\label{fig:searchTermsQuery}
\end{figure}

\begin{figure}[htbp]
\shadowimage[width=\textwidth]{SABIO_manual_8}
\caption{Window with found kinetic equations. The rate equations that match your search criteria are displayed here.
You can select a reaction and click on NEXT or you can click on BACK and change your search criteria.}
\label{fig:foundEquationsManual}
\end{figure}

\begin{figure}[htbp]
\shadowimage[width=\textwidth]{SABIO_manual_9}
\caption{Window for matching elements in the equation to model elements. You can choose how to match the species, compartments, etc., contained in the selected rate 
equation to the respective \SBML elements in your model. The necessary elements to import, such as function definitions contained in the rate equation to add, are also shown.}
\label{fig:matching}
\end{figure}

\begin{figure}[htbp]
\shadowimage[width=\textwidth]{SABIO_manual_10}
\caption{Summary of necessary changes. A summary of all necessary changes to the model for adding the selected rate equation is displayed.
Now click on FINISH and confirm the changes afterwards. Then the rate equation will be added and the necessary changes will be applied to your model.
}
\label{fig:changesManual}
\end{figure}

 

\chapter{Command-line arguments}
\label{chap:CMD}
\renewcommand{\descriptionlabel}[1]{\textcolor{blue}{\texttt{#1}}}

All functions of the program SBMLsqueezer are also available as command-line
arguments. SBMLsqueezer is therefore fully functional even if you do not use
its graphical user interface.
This can be useful if you like to generate kinetic equations, units, and
parameter objects for a large number of \SBML files in a loop or in a more
complex schedule.
To benefit from this functionality, just open a command line window on your
\OS, e.g., excecute the \texttt{cmd.exe} in \Windows or open a terminal window
in \Unix (incl. \Linux and \MacOSX).

It should also be noted that the command line options impact the graphical user
interface. It is hence possible to use command-line options in order to launch
SBMLsqueezer with a pre-defined configuration.

\section{Program usage}

To launch the program from the command line, type
\begin{lstlisting}[language=bash,numbers=none]
java -jar SBMLsqueezer_v2.0.jar org.sbml.squeezer.SBMLsqueezer [options]
\end{lstlisting}
Under \Windows, you might use the command \texttt{javaw} instead of
\texttt{java}. Note that for future versions of SBMLsqueezer the name of the
\JAR could change, because it includes the version number.
Use the following program parameters to lists all available options of
SBMLsqueezer: \texttt{--help, -?}.
The subsequent sections describ all available options of SBMLsqueezer
version~2.0. 

\section{Input/output options}

\begin{description}
\item[--sbml-in-file{[} |={]}<File>]
  Specifies the \SBML input file.
  Default value: none

\item[--sbml-out-file{[} |={]}<File>]
  Specifies the file where SBMLsqueezer writes its \SBML output.
  Default value: none

\item[--try-loading-libsbml]
  If selected, the application will try to load the library \libSBML
  for reading and writing \SBML files, otherwise everything will
  be done with \JSBML only, i.e., pure \Java and therefore platform independent.
  Default value: \texttt{false}
\end{description}

\section{Basic configuration}
With the options introduced in this section, you can specify which assumptions
SBMLsqueezer should make when values or information are missing, i.e., if the 
model is lacking more information than just kinetic equations. You can also
specify how to interprete the model and when error messages should be displayed.

\subsection{General Options}
The behavior of SBMLsqueezer can be specified by the following options, for
instance, to decide if error messages should be displayed under
certain circumstances.
\begin{description}
\item[--overwrite-existing-rate-laws{[} |={]}<Boolean>]
  If this flag is set to \texttt{true}, a new rate law will be created for
  each reaction irrespective of whether there is already a rate
  law assigned to this reaction or not. If \texttt{false} (default), new
  rate laws are only generated if missing in the \SBML file. Note
  that if this option is checked, already existing kinetic laws
  will be overwritten.
  Default value: \texttt{true}

\item[--set-boundary-condition-for-genes{[} |={]}<Boolean>]
  If \texttt{true} (default), the boundary condition of all species that
  represent gene-coding elements, such as genes or gene coding
  regions, will be set to \texttt{true}.
  Default value: \texttt{true}

\item[--all-reactions-as-enzyme-catalyzed{[} |={]}<Boolean>]
  If \texttt{true}, all reactions within the network are considered to be
  enzyme-catalyzed reactions. If \texttt{false} (default), an explicit
  enzymatic catalyst must be assigned to a reaction to obtain
  this status.
  Default value: \texttt{true}

\item[--remove-unnecessary-parameters-and-units{[} |={]}<Boolean>]
  If \texttt{true} (default), parameters and units that are never referenced
  by any element of the model are automatically deleted after
  creating kinetic equations.
  Default value: \texttt{true}

\item[--new-parameters-global]
  If \texttt{true} (default), all parameters are stored globally for the
  whole model. Otherwise the majority of parameters is stored
  locally for the respective kinetic equation they belong to.
  Note that some parameters represent global properties of the
  entire model and should therefore be always stored in the global
  list of prameters. In this way, these parameters are valid within
  the entire model.
  Default value: \texttt{false}

\item[--warnings-for-too-many-reactants{[} |={]}<Boolean>]
  If \texttt{true} (default), warnings will be displayed for reactions with
  an unrealistic number of reactants. The maximal number of reactants
  that are believed to be still realistic can be defined if this
  option is selected.
  Default value: \texttt{true}

\item[--show-sbml-warnings{[} |={]}<Boolean>]
  If \texttt{true} (default), \SBML warnings are displayed. These warnings are
  mainly the result of a syntactical model check and do not give much
  information about the semantic correctness of your model.
  Since the SBML library performs this check, SBMLsqueezer cannot influence the
  content of the syntax check Note that this option only works properly if you
  use \libSBML as your \SBML backend, because \JSBML does not provide a full
  validity check for \SBML models.
  Default value: \texttt{true}

\item[--read-from-sabio-rk{[} |={]}<Boolean>]
  This option lets the user choose whether to search for experimentally
  obtained rate laws in the reaction kinetics database \SABIO. Note that
  performing this search requires an active internet connection.
  Default value: \texttt{true}
\end{description}

\subsection{Default values}

The options in this group allow you to specify several default values to be
applied for components of the model.
This is done in addition to the actual rate law generation, such as the default
compartment size etc., and to define model wide settings to be taken into
account when creating rate equations.
\begin{description}
\item[--max-number-of-reactants{[} |={]}<Integer>]
  A simultaneous collision of a high number of reactants just by
  chance is very unlikely. Usually, these reactions proceed in
  a sequence of separate steps, each involving only very few molecules.
  Here you can specifiy the maximal number of reactants so that
  the reaction is still considered plausible. By default this
  value is set to three. Note that this option is only available if
  you decide that this kind of warning should be displayed.
  Default value: \texttt{3}

\item[--default-compartment-spatial-dim{[} |={]}<Double>]
  If no spatial dimensions are defined for a compartment, the value
  defined by this option will be used as a default.
  Default value: \texttt{3.0}

\item[--default-compartment-size{[} |={]}<Double>]
  For compartments that are not yet initialized, SBMLsqueezer will
  use this value as the default initial size. By default this
  value is set to one.
  Default value: \texttt{1.0}

\item[--default-species-init-val{[} |={]}<Double>]
  If species are not yet initialized, SBMLsqueezer will use this
  value as initial amount or initial concentration of the species.
  Which kind of quantity is used, depends on whether the species
  has only substance units. This means, for species that are to
  be interpreted in terms of concentration, an initial concentration
  will be set, whereas an initial amount will be set if the species
  is to be interpreted in terms of molecule counts. By default
  this value is set to unity.
  Default value: \texttt{1.0}

\item[--default-new-parameter-val{[} |={]}<Double>]
  Here you can specify the default value that is set for newly
  created parameters. By default this value is one.
  Default value: \texttt{1.0}

\item[--default-species-has-only-substance-units{[} |={]}<Boolean>]
  This option allows users to specify that the numerical value
  of a species should be interpreted as a value given in substance
  units in cases where this has not yet been defined. If not selected,
  species with undefined meaning will be conceived as a quantity
  in concentration units.
  Default value: \texttt{true}

\item[--ignore-these-species-when-creating-laws{[} |={]}<String>]
  Allows the user to ignore species that are annotated with the
  given compound identifiers when creating rate laws for reactions
  that involve these species. For instance, water or single protons
  can often be ignored when creating rate equations, hence simplifying
  the resulting rate equations. Preselected are the KEGG compound
  identifiers for several ions and small molecules, including water and
  protons. See \vref{tab:MIRIAMignoreList} for details.
  Default value: \texttt{C00001,} \texttt{C00038,} \texttt{C00070,}
  \texttt{C00076,} \texttt{C00080,} \texttt{C00175,} \texttt{C00238,}
  \texttt{C00282,} \texttt{C00291,} \texttt{C01327,} \texttt{C01528,}
  \texttt{C14818,} \texttt{C14819}
\end{description}

\subsection{Species to be treated as enzymes}

In many situations, it is not clear, which kinds of chemical species can be
considered an enzyme. The following options allow you to select, which kind
of species SBMLsqueezer should interprete as an enzyme when acting as a
catalytic modifier of a reaction.
\begin{description}
\item[--possible-enzyme-antisense-rna{[} |={]}<Boolean>]
  If \texttt{true}, antisense \RNA molecules are treated as enzymes when
  catalyzing a reaction. If \texttt{false} (default) antisense \RNA molecule
  catalyzed reactions are not considered to be enzyme-catalyzed
  reactions.
  Default value: \texttt{false}

\item[--possible-enzyme-complex{[} |={]}<Boolean>]
  If checked (default), complex molecules are treated as enzymes
  when catalyzing a reaction. Otherwise, complex-catalyzed reactions
  are not considered to be enzyme reactions.
  Default value: \texttt{true}

\item[--possible-enzyme-generic{[} |={]}<Boolean>]
  If \texttt{true} (default), generic proteins are treated as enzymes when
  catalyzing a reaction. Otherwise, generic protein-catalyzed
  reactions are not considered to be enzyme reactions.
  Default value: \texttt{true}

\item[--possible-enzyme-macromolecule{[} |={]}<Boolean>]
  If this options is selected, species that are annotated as macromolecules
  are treated as enzymes when catalyzing a reaction. Otherwise,
  macormolecule-catalyzed reactions are not considered enzyme
  reactions. If a modifier of a reaction that is annotated as
  an enzymatic catalyst refers to a macromolecule but this option
  is not active, SBMLsqueezer will reduce the modifier to a simple
  catalyst.
  Default value: \texttt{true}

\item[--possible-enzyme-receptor{[} |={]}<Boolean>]
  If \texttt{true}, receptors are treated as enzymes when catalyzing a reaction.
  If \texttt{false} (default), receptor-catalyzed reactions are not considered
  to be enzyme reactions.
  Default value: \texttt{false}

\item[--possible-enzyme-rna{[} |={]}<Boolean>]
  If \texttt{true} (default), \RNA is treated as an enzyme when catalyzing
  a reaction. Otherwise \RNA-catalyzed reactions are not considered
  to be enzyme-catalyzed reactions.
  Default value: \texttt{true}

\item[--possible-enzyme-simple-molecule]
  If \texttt{true}, simple molecules are treated as enzymes when catalyzing
  a reaction. If \texttt{false} (default), simple molecule-catalyzed reactions
  are not considered to be enzyme reactions.
  Default value: \texttt{false}

\item[--possible-enzyme-truncated{[} |={]}<Boolean>]
  If \texttt{true} (default), truncated proteins are treated as enzymes
  when catalyzing a reaction. Otherwise, truncated protein-catalyzed
  reactions are not considered to be enzyme reactions.
  Default value: \texttt{true}

\item[--possible-enzyme-unknown]
  If \texttt{true}, unknown molecules are treated as enzymes when catalyzing
  a reaction. If \texttt{false} (default), unknown molecule-catalyzed reactions
  are not considered to be enzyme reactions.
  Default value: \texttt{false}
\end{description}

\subsection{How to ensure unit consistency}

Unit consistency is an important property for kinetic models. In \SBML, the
kinetics of all reactions should be defined so that these can be evaluated to
units of substance per time. Since Level~3, the extend units of a model can be
defined. Since then, evaluating reactions should result in extend units per time
units. The main difficulty when dealing with units are transport reactions and
the ability to specify reactive species in terms of molecule counts (amounts) or
concentration units, which is amounts per size. In addition, the sizes of
compartments do not necessarily have to be three dimensional. In \SBML, the
spatial dimensions do not have to be integers either. However, SBMLsqueezer
contains several complex algorithms to determin appropriate units for newly
generated parameters and to incorporate the sizes of surrounding compartments
into the generated rate laws. Here you can specify how to do that.  
\begin{description}
\item[--type-unit-consistency{[} |={]}<UnitConsistencyType>]
  This option ensures unit consistency and can attain two different
  values: Choose \emph{amount} to bring each occurrence of a participating
  species to a substance unit. Depending on whether the species
  has only substance units or not it might be necessary to multiply
  the species with the size of its surrounding compartment. Choose
  \emph{concentration} to bring each participating species to concentration
  units. In this case the species will be divided by the surrounding
  compartment size in a kinetic equations if it is defined to
  have only substance units. The units of parameters are set accordingly.
  All possible values for type \texttt{<UnitConsistencyType>} are:
  \texttt{amount} and \texttt{concentration}.
  Default value: \texttt{amount}
\end{description}

\section{Rate law selection}

The options in this group allow you to define the rate laws with highest
priority when generating kinetic equations in one single step for an entire
network.
To this end, SBMLsqueezer defines several basic types of reactions and provides
a list of applicable generic equations for each type. Special cases of these
equations have to be derived for each individual case, also depending on
unit consistency, compartment dimensions etc.
The actual selection of a rate law means that you can specify, which class of
the program SBMLsqueezer should be used to generate a rate law of this type.
You do not need to have deep programming skills for this. All you need to know
here is that the the names of the rate laws are a bit more complicated than the
actual human-readable names, because you must specify, how SBMLsqueezer
internally names these rate laws. For more advanced uses, it might be
interesting to know that SBMLsqueezer uses the concept known as reflection to
select its kinetic equations.

Furthermore, you can decide how to deal with information about reversible or
irreversible reactions.

\subsection{Reversibility}

These two options are mutually exclusive and their values must not contradict.
Use either one of both.
\begin{description}
\item[--treat-all-reactions-reversible]
  If \texttt{true}, all reactions are set to reversible before creating
  new kinetic equations. Otherwise, the information given by the
  \SBML file will be left unchanged.
  Default value: \texttt{false}

\item[--treat-reactions-reversible-as-given{[} |={]}<Boolean>]
  If checked, the information about reversiblity will be left unchanged.
  Default value: \texttt{true}
\end{description}

\subsection{Gene regulation kinetics}
\begin{description}
\item[--kinetics-gene-regulation{[} |={]}<Class>]
  Please specify the default kinetic law to be applied for reactions
  that are identified to belong to gene-regulatory processes (reactions
  involving genes, \RNA, and proteins), such as transcription or
  translation.
  All possible values for type \texttt{<Class>} are:
  \begin{itemize}
  \item\texttt{org.sbml.squeezer.kinetics.AdditiveModelLinear} (linear additive model, general form),
  \item\texttt{org.sbml.squeezer.kinetics.AdditiveModelNonLinear} (non-linear additive model, general form),
  \item\texttt{org.sbml.squeezer.kinetics.NetGeneratorLinear} (linear additive model, NetGenerator form),
  \item\texttt{org.sbml.squeezer.kinetics.HillEquation} (generalized Hill equation),
  \item\texttt{org.sbml.squeezer.kinetics.HillHinzeEquation} (Hill-Hinze equation), 
  \item\texttt{org.sbml.squeezer.kinetics.HillRaddeEquation} (Hill-Radde equation),
  \item\texttt{org.sbml.squeezer.kinetics.HSystem} (H-system equation by \citealp{Spieth2006}),
  \item\texttt{org.sbml.squeezer.kinetics.NetGeneratorNonLinear} (non-linear additive model, NetGenerator form),
  \item\texttt{org.sbml.squeezer.kinetics.SSystem} (S-System-based kinetic),
  \item\texttt{org.sbml.squeezer.kinetics.Vohradsky} (Non-linear additive model by \citealp*{Vu2007}), and
  \item\texttt{org.sbml.squeezer.kinetics.Weaver} (non-linear additive model by \citealp{Weaver1999}).
  \end{itemize}
  Default value: \texttt{org.sbml.squeezer.kinetics.HillHinzeEquation}

\item[--kinetics-zero-reactants{[} |={]}<Class>]
  Default rate law with zeroth order reactants
  All possible values for type \texttt{<Class>} are:
  \begin{itemize}
  \item\texttt{org.sbml.squeezer.kinetics.AdditiveModelLinear} (linear additive model, general form),
  \item\texttt{org.sbml.squeezer.kinetics.AdditiveModelNonLinear} (non-linear additive model, general form),
  \item\texttt{org.sbml.squeezer.kinetics.HillHinzeEquation} (Hill-Hinze equation), 
  \item\texttt{org.sbml.squeezer.kinetics.HillRaddeEquation} (Hill-Radde equation),
  \item\texttt{org.sbml.squeezer.kinetics.HSystem} (H-system equation by \citealp{Spieth2006}),
  \item\texttt{org.sbml.squeezer.kinetics.NetGeneratorLinear} (linear additive model, NetGenerator form),
  \item\texttt{org.sbml.squeezer.kinetics.NetGeneratorNonLinear} (non-linear additive model, NetGenerator form),
  \item\texttt{org.sbml.squeezer.kinetics.SSystem} (S-System-based kinetic),
  \item\texttt{org.sbml.squeezer.kinetics.Vohradsky} (Non-linear additive model by \citealp*{Vu2007}),
  \item\texttt{org.sbml.squeezer.kinetics.Weaver} (non-linear additive model by \citealp{Weaver1999}),
  \item\texttt{org.sbml.squeezer.kinetics.ZerothOrderForwardGMAK} (zeroth order forward mass action kinetics), and
  \item\texttt{org.sbml.squeezer.kinetics.ZerothOrderReverseGMAK} (zeroth order reverse mass action kinetics).
  \end{itemize}
  Default value: \texttt{org.sbml.squeezer.kinetics.ZerothOrderReverseGMAK}

\item[--kinetics-zero-products{[} |={]}<Class>]
  Default rate law with zeroth order products
  All possible values for type \texttt{<Class>} are:
  \begin{itemize}
  \item\texttt{org.sbml.squeezer.kinetics.AdditiveModelLinear} (linear additive model, general form),
  \item\texttt{org.sbml.squeezer.kinetics.AdditiveModelNonLinear} (non-linear additive model, general form),
  \item\texttt{org.sbml.squeezer.kinetics.HillHinzeEquation} (Hill-Hinze equation),
  \item\texttt{org.sbml.squeezer.kinetics.HillRaddeEquation} (Hill-Radde equation),
  \item\texttt{org.sbml.squeezer.kinetics.HSystem} (Hill-Radde equation),
  \item\texttt{org.sbml.squeezer.kinetics.NetGeneratorLinear} (linear additive model, NetGenerator form),
  \item\texttt{org.sbml.squeezer.kinetics.NetGeneratorNonLinear} (non-linear additive model, NetGenerator form),
  \item\texttt{org.sbml.squeezer.kinetics.SSystem} (S-System-based kinetic),
  \item\texttt{org.sbml.squeezer.kinetics.Vohradsky} (Non-linear additive model by \citealp*{Vu2007}),
  \item\texttt{org.sbml.squeezer.kinetics.Weaver} (non-linear additive model by \citealp{Weaver1999}),
  \item\texttt{org.sbml.squeezer.kinetics.ZerothOrderForwardGMAK} (zeroth order forward mass action kinetics), and
  \item\texttt{org.sbml.squeezer.kinetics.ZerothOrderReverseGMAK} (zeroth order reverse mass action kinetics).
  \end{itemize}
  Default value: \texttt{org.sbml.squeezer.kinetics.ZerothOrderReverseGMAK}
\end{description}

\subsection{Reversible rate laws}
\begin{description}
\item[--type-standard-version{[} |={]}<TypeStandardVersion>]
  This option declares the version of the modular rate laws and
  can attain the three different values \emph{cat}, \emph{hal}, and \emph{weg}
  as described in the publications of \citet{Liebermeister2010}. This option
  can only be accessed if all reactions are modeled reversibly.
  All possible values for type \texttt{<TypeStandardVersion>} are:
  \texttt{cat}, \texttt{hal}, and \texttt{weg}.
  Default value: \texttt{cat}

\item[--kinetics-reversible-non-enzyme-reactions{[} |={]}<Class>]
  Determines the key for the standard kinetic law to be applied
  for reactions that are catalyzed by non-enzymes or that are
  not catalyzed at all. The value may be any rate law that implements
  \texttt{InterfaceNonEnzymeKinetics}
  All possible values for type \texttt{<Class>} are:
  \begin{itemize}
  \item\texttt{org.sbml.squeezer.kinetics.GeneralizedMassAction} (generalized mass-action),
  \item\texttt{org.sbml.squeezer.kinetics.ZerothOrderForwardGMAK} (zeroth order forward mass action kinetics), and
  \item\texttt{org.sbml.squeezer.kinetics.ZerothOrderReverseGMAK} (zeroth order reverse mass action kinetics).
  \end{itemize}
  Default value: \texttt{org.sbml.squeezer.kinetics.GeneralizedMassAction}

\item[--kinetics-reversible-uni-uni-type{[} |={]}<Class>]
  This key defines the default kinetic law to be applied to enzyme-catalyzed
  reactions with one reactant and one product.
  All possible values for type \texttt{<Class>} are:
  \begin{itemize}
  \item\texttt{org.sbml.squeezer.kinetics.CommonModularRateLaw} (common modular rate law, CM),
  \item\texttt{org.sbml.squeezer.kinetics.ConvenienceKinetics} (convenience kinetics),
  \item\texttt{org.sbml.squeezer.kinetics.DirectBindingModularRateLaw} (direct binding modular rate law, DM),
  \item\texttt{org.sbml.squeezer.kinetics.ForceDependentModularRateLaw} (force-dependent modular rate law, FM),
  \item\texttt{org.sbml.squeezer.kinetics.HillEquation} (generalized Hill equation),
  \item\texttt{org.sbml.squeezer.kinetics.MichaelisMenten} (Michaelis-Menten),
  \item\texttt{org.sbml.squeezer.kinetics.PowerLawModularRateLaw} (power-law modular rate law, PM), and
  \item\texttt{org.sbml.squeezer.kinetics.SimultaneousBindingModularRateLaw} (simultaneous binding modular rate law, SM).
  \end{itemize}
  Default value: \texttt{org.sbml.squeezer.kinetics.MichaelisMenten}

\item[--kinetics-reversible-bi-uni-type{[} |={]}<Class>]
  Choose the type of the default kinetic law for reversible bi-uni
  reactions (two reactants, one product).
  All possible values for type \texttt{<Class>} are:
  \begin{itemize}
  \item\texttt{org.sbml.squeezer.kinetics.CommonModularRateLaw} (common modular rate law, CM),
  \item\texttt{org.sbml.squeezer.kinetics.ConvenienceKinetics} (convenience kinetics),
  \item\texttt{org.sbml.squeezer.kinetics.DirectBindingModularRateLaw} (direct binding modular rate law, DM),
  \item\texttt{org.sbml.squeezer.kinetics.ForceDependentModularRateLaw} (force-dependent modular rate law, FM),
  \item\texttt{org.sbml.squeezer.kinetics.OrderedMechanism} (ordered mechanism),
  \item\texttt{org.sbml.squeezer.kinetics.PowerLawModularRateLaw} (power-law modular rate law, PM),
  \item\texttt{org.sbml.squeezer.kinetics.RandomOrderMechanism} (random order mechanism), and
  \item\texttt{org.sbml.squeezer.kinetics.SimultaneousBindingModularRateLaw} (simultaneous binding modular rate law, SM).
  \end{itemize}
  Default value: \texttt{org.sbml.squeezer.kinetics.RandomOrderMechanism}

\item[--kinetics-reversible-bi-bi-type{[} |={]}<Class>]
  Select the type of the default kinetic law for reversible bi-bi
  reactions (two reactants, two products).
  All possible values for type \texttt{<Class>} are:
  \begin{itemize}
  \item\texttt{org.sbml.squeezer.kinetics.CommonModularRateLaw} (common modular rate law, CM),
  \item\texttt{org.sbml.squeezer.kinetics.ConvenienceKinetics} (convenience kinetics),
  \item\texttt{org.sbml.squeezer.kinetics.DirectBindingModularRateLaw} (direct binding modular rate law, DM),
  \item\texttt{org.sbml.squeezer.kinetics.ForceDependentModularRateLaw} (force-dependent modular rate law, FM),
  \item\texttt{org.sbml.squeezer.kinetics.OrderedMechanism} (ordered mechanism),
  \item\texttt{org.sbml.squeezer.kinetics.PingPongMechanism} (Ping-Pong mechanism),
  \item\texttt{org.sbml.squeezer.kinetics.PowerLawModularRateLaw} (power-law modular rate law, PM),
  \item\texttt{org.sbml.squeezer.kinetics.RandomOrderMechanism} (random order mechanism), and
  \item\texttt{org.sbml.squeezer.kinetics.SimultaneousBindingModularRateLaw} (simultaneous binding modular rate law, SM).
  \end{itemize}
  Default value: \texttt{org.sbml.squeezer.kinetics.RandomOrderMechanism}

\item[--kinetics-reversible-arbitrary-enzyme-reactions{[} |={]}<Class>]
  Arbitrary reversible enzyme reactions
  All possible values for type \texttt{<Class>} are:
  \begin{itemize}
  \item\texttt{org.sbml.squeezer.kinetics.CommonModularRateLaw} (common modular rate law, CM),
  \item\texttt{org.sbml.squeezer.kinetics.ConvenienceKinetics} (convenience kinetics),
  \item\texttt{org.sbml.squeezer.kinetics.DirectBindingModularRateLaw} (direct binding modular rate law, DM),
  \item\texttt{org.sbml.squeezer.kinetics.ForceDependentModularRateLaw} (force-dependent modular rate law, FM),
  \item\texttt{org.sbml.squeezer.kinetics.PowerLawModularRateLaw} (power-law modular rate law, PM), and
  \item\texttt{org.sbml.squeezer.kinetics.SimultaneousBindingModularRateLaw} (simultaneous binding modular rate law, SM).
  \end{itemize}
  Default value: \texttt{org.sbml.squeezer.kinetics.CommonModularRateLaw}
\end{description}

\subsection{Irreversible rate laws}
\begin{description}
\item[--kinetics-irreversible-non-enzyme-reactions{[} |={]}<Class>]
  Determines the key for the standard kinetic law to be applied
  for reactions that are catalyzed by non-enzymes or that are
  not catalyzed at all. The value may be any rate law that implements the
  interface
  \texttt{org.sbml.squeezer.kinetics.InterfaceNonEnzymeKinetics}
  All possible values for type \texttt{<Class>} are:
  \begin{itemize}
  \item\texttt{org.sbml.squeezer.kinetics.GeneralizedMassAction} (generalized mass-action),
  \item\texttt{org.sbml.squeezer.kinetics.ZerothOrderForwardGMAK} (zeroth order forward mass action kinetics), and
  \item\texttt{org.sbml.squeezer.kinetics.ZerothOrderReverseGMAK} (zeroth order reverse mass action kinetics).
  \end{itemize}
  Default value: \texttt{org.sbml.squeezer.kinetics.GeneralizedMassAction}

\item[--kinetics-irreversible-uni-uni-type{[} |={]}<Class>]
  This key defines the default kinetic law to be applied to enzyme-catalyzed
  reactions with one reactant and one product.
  All possible values for type \texttt{<Class>} are:
  \begin{itemize}
  \item\texttt{org.sbml.squeezer.kinetics.ConvenienceKinetics} (convenience kinetics),
  \item\texttt{org.sbml.squeezer.kinetics.IrrevCompetNonCooperativeEnzymes} (irreversible non-exclusive non-cooperative competitive inihibition),
  \item\texttt{org.sbml.squeezer.kinetics.IrrevNonModulatedNonInteractingEnzymes} (irreversible non-modulated non-interacting reactant enzymes),
  \item\texttt{org.sbml.squeezer.kinetics.HillEquation} (generalized Hill equation), and
  \item\texttt{org.sbml.squeezer.kinetics.MichaelisMenten} (Michaelis-Menten).
  \end{itemize}
  Default value: \texttt{org.sbml.squeezer.kinetics.MichaelisMenten}

\item[--kinetics-irreversible-bi-uni-type{[} |={]}<Class>]
  Choose the type of the default kinetic law for irreversible bi-uni
  reactions (two reactants, one product).
  All possible values for type \texttt{<Class>} are:
  \begin{itemize}
  \item\texttt{org.sbml.squeezer.kinetics.ConvenienceKinetics} (convenience kinetics),
  \item\texttt{org.sbml.squeezer.kinetics.IrrevNonModulatedNonInteractingEnzymes} (irreversible non-modulated non-interacting reactant enzymes),
  \item\texttt{org.sbml.squeezer.kinetics.OrderedMechanism} (ordered mechanism), and
  \item\texttt{org.sbml.squeezer.kinetics.RandomOrderMechanism} (random order mechanism).
  \end{itemize}
  Default value: \texttt{org.sbml.squeezer.kinetics.RandomOrderMechanism}

\item[--kinetics-irreversible-bi-bi-type{[} |={]}<Class>]
  Select the type of the default kinetic law for irreversible bi-bi
  reactions (two reactants, two products).
  All possible values for type \texttt{<Class>} are:
  \begin{itemize}
  \item\texttt{org.sbml.squeezer.kinetics.ConvenienceKinetics} (convenience kinetics),
  \item\texttt{org.sbml.squeezer.kinetics.IrrevNonModulatedNonInteractingEnzymes} (irreversible non-modulated non-interacting reactant enzymes),
  \item\texttt{org.sbml.squeezer.kinetics.OrderedMechanism} (ordered mechanism),
  \item\texttt{org.sbml.squeezer.kinetics.PingPongMechanism} (Ping-Pong mechanism), and
  \item\texttt{org.sbml.squeezer.kinetics.RandomOrderMechanism} (random order mechanism).
  \end{itemize}
  Default value: \texttt{org.sbml.squeezer.kinetics.RandomOrderMechanism}

\item[--kinetics-irreversible-arbitrary-enzyme-reactions{[} |={]}<Class>]
  Arbitrary irreversible enzyme reactions
  All possible values for type \texttt{<Class>} are:
  \begin{itemize}
  \item\texttt{org.sbml.squeezer.kinetics.ConvenienceKinetics} (convenience kinetics) and
  \item\texttt{org.sbml.squeezer.kinetics.IrrevNonModulatedNonInteractingEnzymes} (irreversible non-modulated non-interacting reactant enzymes).
  \end{itemize}
  Default value:
  \texttt{org.sbml.squeezer.kinetics.IrrevNonModulatedNonInteractingEnzymes}
\end{description}

\section{\SABIO search preferences}

For querying the content of the \SABIO database for experimentally determined
kinetic equations, SBMLsqueezer provides a large set of options in order to
customize your search.

\subsection{General properties}
\begin{description}
\item[--is-wildtype{[} |={]}<Boolean>]
  Search for wildtype kinetics.
  Default value: \texttt{true}

\item[--is-mutant{[} |={]}<Boolean>]
  Search for kinetics of mutants.
  Default value: \texttt{true}

\item[--is-recombinant]
  Search for kinetics of recombinant organisms.
  Default value: \texttt{false}

\item[--has-kinetic-data{[} |={]}<Boolean>]
  Search for entries containing kinetic data.
  Default value: \texttt{true}

\item[--is-direct-submission{[} |={]}<Boolean>]
  Search for entries directly submitted.
  Default value: \texttt{true}

\item[--is-journal{[} |={]}<Boolean>]
  Search for entries referring to journal publications.
  Default value: \texttt{true}

\item[--is-entries-inserted-since]
  Consider only entries inserted after the specified date.
  Default value: \texttt{false}
\end{description}

\subsection{Temperature}
\begin{description}
\item[--lowest-temperature-value{[} |={]}<Double>]
  The lowest possible temperature for entries (in $^\circ$C).
  Arguments must fit into the range {[-271.15, 1000]}.
  Default value: \texttt{-10.0}

\item[--highest-temperature-value{[} |={]}<Double>]
  The highest possible temperature for entries (in $^\circ$C).
  Arguments must fit into the range {[}-271.15, 1000{]}.
  Default value: \texttt{115.0}
\end{description}

\subsection{Range of pH values}
\begin{description}
\item[--lowest-ph-value{[} |={]}<Double>]
  The lowest possible pH value for entries.
  Arguments must fit into the range {[}0, 14{]}.
  Default value: \texttt{0.0}

\item[--highest-ph-value{[} |={]}<Double>]
  The highest possible pH value for entries.
  Arguments must fit into the range {[}0, 14{]}.
  Default value: \texttt{14.0}
\end{description}

\subsection{Date}
\begin{description}
\item[--lowest-date{[} |={]}<Date>]
  Define the earliest acceptable date when the entries have been
  inserted into \SABIO.
  Default value: \texttt{Wed Oct 15 00:00:00 PDT 2008}
\end{description}

\section{\SABIO search options}
\subsection{General options}
\begin{description}
\item[--pathway{[} |={]}<String>]
  Define the pathway for which the kinetics are to be determined.

\item[--tissue{[} |={]}<String>]
  Define the tissue for which the kinetics are to be determined.

\item[--cellular-location{[} |={]}<String>]
  Define the cellular location for which the kinetics are to be
  determined.

\item[--organism{[} |={]}<String>]
  Define the organism for which the kinetics are to be determined.
\end{description}

\section{Options for the graphical user interface}

The options in this group allow you to influence the behavior of the graphical
user interface. This means that you can directly launch SBMLsqueezer with your
preferred configuration.

\subsection{Additional options}
\begin{description}
\item[--check-for-updates{[} |={]}<Boolean>]
  Decide whether or not this program should search for updates
  at start-up.
  Default value: \texttt{true}

\item[--gui] If this option is given, the program will display its graphical
  user interface.
  Default value: \texttt{false}

\item[--log-level{[} |={]}<String>]
  Change the log-level of this application. This option will influence how
  fine-grained error and other log messages will be that you receive while
  executing this program.
  All possible values for type \texttt{<String>} are:
  \begin{itemize}
  \item \texttt{ALL}: all log messages will be displayed,
  \item \texttt{CONFIG}: those log messages related to configuration of the program and less fine-grained will be displayed,
  \item \texttt{FINE}:  displays simple debugging messages and less fine-grained messages,
  \item \texttt{FINER}: the messages that will be displayed are already relevant for more extensive debugging purposes and less fine-grained messages,
  \item \texttt{FINEST}: information relevant for intensive debugging the program and less fine-grained will be displayed,
  \item \texttt{INFO}: information messages and less fine-grained messages, such as warning messages, will be displayed,
  \item \texttt{OFF}: no log messages will be displayed at all,
  \item \texttt{SEVERE}: only serious error messages will be displayed, and
  \item \texttt{WARNING}: only warnings and serious error messages will be displayed.
  \end{itemize}
  Default value: \texttt{INFO}
\end{description}

\section{LaTeX Options}

The program SBMLsqueezer brings with it a full version of the latest development
release of \SBMLLaTeX. Hence, all options provided by this report generator for
\SBML models can also be applied to SBMLsqueezer. For more information, see
the project web site of
\SBMLLaTeX\footnote{\url{http://www.cogsys.cs.uni-tuebingen.de/software/SBML2LaTeX/}}
and the corresponding publication \citep{Draeger2009b}.

\subsection{LaTeX compiler location}
\begin{description}
\item[--load-latex-compiler{[} |={]}<File>]
  The path to the \LaTeX{} compiler to generate PDF, DVI or other
  files from the created \LaTeX{} report file. Accepts all files
  (*).
  Default value: \texttt{<current directory>}
\end{description}

\subsection{Report options}
\begin{description}
\item[--check-consistency]
  If \texttt{true}, the automatic model consistency check is performed and
  the results are written in the appendix of the model report file.
  Note that this might require an active internet connection.
  Default value: \texttt{false}

\item[--miriam-annotation{[} |={]}<Boolean>]
  If \texttt{true} (default), \MIRIAM annotations are included into the model
  report if there are any. In this case, \SBMLLaTeX generates
  links to the resources for each annotated element.
  Default value: \texttt{true}

\item[--show-predefined-units{[} |={]}<Boolean>]
  If \texttt{true} (default), all predefined unit declarations of the \SBML
  are made explicit in the report file as these are defined by
  the corresponding \SBML Level and Version. Otherwise only unit
  definitions from the model are included. Note that this option
  is only available if the option \texttt{--include-section-unit-definitions}
  is active.
  Default value: \texttt{true}

\item[--print-full-ode-system]
  If set to \texttt{true}, the entire rate of change will be written for
  each species. By default, \SBMLLaTeX only prints the sum of
  the individual reaction rates, which are hyper-linked but displayed
  at a different position of the report. Note that this option
  is only available if the option \texttt{--include-section-reactions}
  is active.
  Default value: \texttt{false}

\item[--clean-workspace]
  If this option is set to \texttt{true}, all temporary files will be deleted
  after running \SBMLLaTeX. In case of PDF creation, for instance,
  this will cause even the \TeX{} file to be deleted. However, this
  option can be meaningful to remove all the temporary files created
  by your system's \LaTeX{} compiler.
  Default value: \texttt{false}
\end{description}

\subsection{Layout options}
\begin{description}
\item[--landscape]
  This option decides whether to set the \LaTeX{} document in landscape
  or portrait mode. By default most pages are in portrait format.
  Default value: \texttt{false}

\item[--print-names-if-available]
  If selected, the names of \SBML elements (\texttt{NamedSBase}) are displayed
  instead of their identifiers. This can only be done if the element
  has a name.
  Default value: \texttt{false}

\item[--title-page]
  If \texttt{true}, a separate title page will be created. By default the
  title is written as a simple heading on the first page.
  Default value: \texttt{false}

\item[--typewriter{[} |={]}<Boolean>]
  This option decides whether a typewriter font should be applied
  to highlight \SBML identifiers. This is particularly important
  when these occur in mathematical equations.
  Default value: \texttt{true}

\item[--reactants-overview-table]
  If \texttt{true}, the details (identifier and name) of all reactants,
  modifiers and products participating in a reaction are listed
  in one table. By default a separate table is created for each
  one of the three participant groups including its \SBO term.
  Note that this option is only available if the option
  \texttt{--include-section-reactions} is active.
  Default value: \texttt{false}
\end{description}

\subsection{Typographical options}
\begin{description}
\item[--font-headings{[} |={]}<SansSerifFont>]
  Allows to select the font of captions and other (by default sans
  serif) text.
  All possible values for type \texttt{<SansSerifFont>} are:
  \begin{itemize}
  \item \texttt{avant} {\fontfamily{pag}\selectfont (sample in Avant Garde)},
  \item \texttt{cmss} {\fontfamily{cmss}\selectfont (sample in Computer Modern Sans Serif)}, and
  \item \texttt{helvetica} {\fontfamily{phv}\selectfont (sample in Helvetica)}.
  \end{itemize}
  Default value: \texttt{helvetica}

\item[--font-size{[} |={]}<Short>]
  This option allows you to select the size of the standard text
  font. Headings appear with a larger font.
  All possible values for type \texttt{<Short>} are:
  \texttt{8}, \texttt{9}, \texttt{10}, \texttt{11}, \texttt{12}, \texttt{14},
  and \texttt{17}.
  Default value: \texttt{11}

\item[--font-text{[} |={]}<SerifFont>]
  Allows to select the font of continuous text. Choosing \texttt{times}
  is actually not recommended because in some cases equations
  might not look as nicely as they do when using \texttt{mathptmx}.
  All possible values for type \texttt{<SerifFont>} are:
  \begin{itemize}
  \item \texttt{chancery} {\fontfamily{pzc}\selectfont (sample in Chancery)},
  \item \texttt{charter}  {\fontfamily{bch}\selectfont (sample in Charter)},
  \item \texttt{cmr}  {\fontfamily{cmr}\selectfont (sample in Computer Modern Roman)},
  \item \texttt{mathptmx}  {\fontfamily{ptm}\selectfont (sample in Times for math)},
  \item \texttt{palatino}  {\fontfamily{ppl}\selectfont (sample in Palatino)},
  \item \texttt{times}  {\fontfamily{ptm}\selectfont (sample in Times)}, and
  \item \texttt{utopia}  {\fontfamily{put}\selectfont (sample in Utopia)}.
  \end{itemize}
  Default value: \texttt{mathptmx}

\item[--font-typewriter{[} |={]}<String>]
  Select a typewriter font that can be used for identifiers if
  option 'TYPEWRITER' is selected. URLs and other resources are
  also marked with this font.
  All possible values for type \texttt{<String>} are:
  \begin{itemize}
  \item \texttt{cmt} {\fontfamily{cmt}\selectfont (sample in Computer Modern Typewriter)} and
  \item \texttt{courier} {\fontfamily{pcr}\selectfont (sample in Courier)}.
  \end{itemize}
  Default value: \texttt{cmt}

\item[--paper-size{[} |={]}<PaperSize>]
  The paper size for \LaTeX{} documents. With this option the paper
  format can be influenced. Default paper size: DIN A4. All sizes
  a?, b?, c? and d? are European DIN sizes. Letter, legal and
  executive are US paper formats.
  All possible values for type \texttt{<PaperSize>} are:
  \texttt{letter}, \texttt{legal},
  \texttt{executive}, \texttt{a0},
  \texttt{a1}, \texttt{a2},
  \texttt{a3}, \texttt{a4},
  \texttt{a5}, \texttt{a6},
  \texttt{a7}, \texttt{a8},
  \texttt{a9}, \texttt{b0},
  \texttt{b1}, \texttt{b2},
  \texttt{b3}, \texttt{b4},
  \texttt{b5}, \texttt{b6},
  \texttt{b7}, \texttt{b8},
  \texttt{b9}, \texttt{c0},
  \texttt{c1}, \texttt{c2},
  \texttt{c3}, \texttt{c4},
  \texttt{c5}, \texttt{c6},
  \texttt{c7}, \texttt{c8},
  \texttt{c9}, \texttt{d0},
  \texttt{d1}, \texttt{d2},
  \texttt{d3}, \texttt{d4},
  \texttt{d5}, \texttt{d6},
  \texttt{d7}, \texttt{d8},
  and \texttt{d9}.
  Default value: \texttt{letter}
\end{description}

\subsection{Content of the report}
\begin{description}
\item[--include-section-compartment-types{[} |={]}<Boolean>]
  This option decides whether or not a section about compartment
  types should be included in the resulting model report. Note
  that this option only causes an effect if the model contains
  compartment type declarations.
  Default value: \texttt{true}

\item[--include-section-compartments{[} |={]}<Boolean>]
  This option decides whether or not a section about compartments
  should be included in the resulting model report. Note that
  this option only causes an effect if the model contains compartment
  declarations.
  Default value: \texttt{true}

\item[--include-section-constraints{[} |={]}<Boolean>]
  This option decides whether or not a section about constraints
  should be included in the resulting model report. Note that
  this option only causes an effect if the model contains constraint
  declarations.
  Default value: \texttt{true}

\item[--include-section-events{[} |={]}<Boolean>]
  This option decides whether or not a section about events should
  be included in the resulting model report. Note that this option
  only causes an effect if the model contains event declarations.
  Default value: \texttt{true}

\item[--include-section-function-definitions{[} |={]}<Boolean>]
  This option decides whether or not a section about function definitions
  should be included in the resulting model report. Note that
  this option only causes an effect if the model declares any
  function definitions.
  Default value: \texttt{true}

\item[--include-section-initial-assignments{[} |={]}<Boolean>]
  This option decides whether or not a section about initial assignments
  should be included in the resulting model report. Note that
  this option only causes an effect if the model declares any
  initial assignments.
  Default value: \texttt{true}

\item[--include-section-parameters{[} |={]}<Boolean>]
  This option decides whether or not a section about parameters
  should be included in the resulting model report. Note that
  this option only causes an effect if the model declares any
  parameters.
  Default value: \texttt{true}

\item[--include-section-reactions{[} |={]}<Boolean>]
  This option decides whether or not a section about reactions
  should be included in the resulting model report. Note that
  this option only causes an effect if the model declares any
  reactions. Furthermore, this option also decides if a summary
  of the differential equation system that is implied by the given
  model should be generated. Again, this will only cause an effect
  if the model contains any species.
  Default value: \texttt{true}

\item[--include-section-rules{[} |={]}<Boolean>]
  This option decides whether or not a section about rules should
  be included in the resulting model report. Note that this option
  only causes an effect if the model declares any rules, no matter
  if these are of algebraic, assignment or rate rule type.
  Default value: \texttt{true}

\item[--include-section-species{[} |={]}<Boolean>]
  This option decides whether or not a section about the species
  in the given model should be included in the resulting model
  report. Note that this option only causes an effect if the model
  declares any species.
  Default value: \texttt{true}

\item[--include-section-species-types{[} |={]}<Boolean>]
  If this option is selected, a section about species types will
  occur in the model report. Otherwise, this section will be excluded
  from the report.
  Default value: \texttt{true}

\item[--include-section-unit-definitions{[} |={]}<Boolean>]
  This option decides whether or not a section about the unit definitions
  of the given model should be included in the resulting model
  report. Note that this option only causes an effect if the model
  declares any unit definitions. However, in some level/version
  combinations \SBML models contain predefined unit definitions
  which might be included in the model report if this option is
  active.
  Default value: \texttt{true}
\end{description}

\subsection{Additional options}
\begin{description}
\item[--include-section-layouts{[} |={]}<Boolean>]
  Include a section with images of model layouts if these are available.
  Default value: \texttt{true}
\end{description}

\section{\Garuda options}

\Garuda is a software framework that allows multiple applications (gadgets) to
communicate with each other by sharing data files. In addition, you can launch
an application from \Garuda, query for tools with specific aims and much more.
Since version 2.0, SBMLsqueezer can also be used as a gadget in \Garuda.

\begin{description}
\item[--connect-to-garuda{[} |={]}<Boolean>]
  Decides whether or not the current application should attempt to connect to
  the \Garuda Core. Default value: \texttt{true}
\end{description}
\renewcommand{\descriptionlabel}[1]{\textcolor{black}{\textbf{#1}}}

\chapter{Supported rate laws}\label{chap:RateLaws}

The kinds of equations supported by the program are
numerous, including traditional approaches \citep{Guldberg1879, Michaelis1913}
just like very recent equations \citep{Liebermeister2006, Liebermeister2010}.
It provides equations for gene-regulatory processes
\citep{Hinze2007, Radde2007a, Toepfer2007, Vu2007,Weaver1999} and approximative
rate laws \citep{Savageau1969}.
In addition, SBMLsqueezer covers a large variety of standard rate laws for
biochemical reactions from relevant text books
\citep{Segel1993, Heinrich1996, Bisswanger2000, Cornish-Bowden2004}.
Here, we give a short overview of all equations that are currently implemented
in SBMLsqueezer, ordered by the categories metabolic or gene-regulatory.
However, the actual algorithm that suggests applicable rate laws is more complex
and considers several features of the reaction.
It can therefore happen that SBMLsqueezer suggests multiple different rate
equations for the same reaction.

\section{Rate laws for metabolic processes}
\begin{itemize}
  \item (Generalized) mass-action rate law with numerous orders \citep[p.~16]{Guldberg1879, Heinrich1996}
  \item Uni-uni Michaelis-Menten kinetics \citep{Michaelis1913}
  \item Irreversible non-modulated non-interacting reactant enzymes (\SBO)
  \item Bi-uni enzyme mechanisms \citep{Segel1993, Bisswanger2000, Cornish-Bowden2004}
  \begin{itemize}
    \item Random-order mechanism
    \item Ordered mechanism
  \end{itemize}
  \item Bi-bi enzyme reactions \citep{Segel1993, Bisswanger2000, Cornish-Bowden2004}
  \begin{itemize}
    \item Random-order mechanism \citep[p.~169]{Cornish-Bowden2004}
    \item Ordered mechanism
    \item Ping-pong mechanism
  \end{itemize}
  \item Modular rate laws for enzymati reactions \citep{Liebermeister2010}
  \begin{itemize}
    \item Power-law modular rate law (PM)
    \item Common modular rate law (CM)
    \item Direct binding modular rate law (DM)
    \item Simultaneous binding modular rate law (SM)
    \item Force-dependent modular rate law (FM)
  \end{itemize}
  \item Convenience kinetics \citep{Liebermeister2006}
  \begin{itemize}
    \item Thermodynamically dependent form
    \item Thermodynamically independent form
  \end{itemize}
  \item (Generalized) Hill equation \citep[p.~314]{Hill1910, Cornish-Bowden2004}
\end{itemize}

\section{Rate laws for gene-regulatory processes}
\begin{itemize}
  \item Hill-Hinze equation \citep{Hinze2007}
  \item Hill-Radde equation \citep{Radde2007a, Radde2007}
  \item Linear additive network models
    \begin{itemize}
      \item General form
      \item NetGenerator form \citep{Toepfer2007}
    \end{itemize}
  \item Non-linear additive network models
    \begin{itemize}
      \item General form
      \item NetGenerator form \citep{Toepfer2007}
      \item Vohradsk{\'y}'s equation \citep{Vu2007}
      \item Weaver's equation \citep{Weaver1999}
    \end{itemize}
  \item S-systems \citep{Savageau1969, spieth04optimizing, Tournier2005, Spieth2006, Hecker2009}
  \item H-systems \citep{Spieth2006}
\end{itemize}

\begin{table}[htb]
\centering
\caption[KEGG identifiers of small molecules and ions]{KEGG identifiers of
small molecules and ions. This table gives the default list of all small
molecules and ions that are ignored by SBMLsqueezer when creating kinetic
equations. This list was created according to \citet{Blum2009}.}
\label{tab:MIRIAMignoreList}
\begin{tabular}{lll}
\toprule
Chemical formula & Common name & KEGG identifier \\
\midrule
\ce{H2O} & Water &
\href{http://identifiers.org/kegg.compound/C00001}{\texttt{C00001}}\\
\ce{Zn^{2+}} & Zinc ion &
\href{http://identifiers.org/kegg.compound/C00038}{\texttt{C00038}}\\
\ce{Cu^{2+}} & Copper ion &
\href{http://identifiers.org/kegg.compound/C00070}{\texttt{C00070}}\\
\ce{Ca^{2+}} & Calcium ion &
\href{http://identifiers.org/kegg.compound/C00076}{\texttt{C00076}}\\
\ce{H+} & Proton &
\href{http://identifiers.org/kegg.compound/C00080}{\texttt{C00080}}\\
\ce{Co^{2+}} & Cobalt ion &
\href{http://identifiers.org/kegg.compound/C00175}{\texttt{C00175}}\\
\ce{K+} & Potassium ion &
\href{http://identifiers.org/kegg.compound/C00238}{\texttt{C00238}}\\
\ce{H2} & Hydrogen &
\href{http://identifiers.org/kegg.compound/C00282}{\texttt{C00282}}\\
\ce{Ni^{2+}} & Nickel ion &
\href{http://identifiers.org/kegg.compound/C00291}{\texttt{C00291}}\\
\ce{Cl-} & Chloride ion &
\href{http://identifiers.org/kegg.compound/C00698}{\texttt{C00698}}\\
\ce{HCl} & Hydrochloric acid &
\href{http://identifiers.org/kegg.compound/C01327}{\texttt{C01327}}\\
\ce{H2Se} & Hydrogen selenide &
\href{http://identifiers.org/kegg.compound/C01528}{\texttt{C01528}}\\
\ce{Fe^{2+}} & Iron II ion &
\href{http://identifiers.org/kegg.compound/C14818}{\texttt{C14818}}\\
\ce{Fe^{3+}} & Iron III ion &
\href{http://identifiers.org/kegg.compound/C14819}{\texttt{C14819}}\\
\bottomrule
\end{tabular}
\end{table}


% 03_Troubleshooting
\chapter{FAQ and troubleshooting}
\label{ch:faq}

TODO: These are some template questions. Add new ones and modify/ remove the
old ones to suit your needs.

\noindent \textbf{Where can I get help for a certain component/ option/ checkbox/ etc.?}\newline
Most elements in SBMLsqueezer have tooltips. If you don't understand an option, you
can get help in the first place by just pointing the mouse cursor over it and
wait for the tooltip to show up ($\sim$ 3 seconds).\newline

\noindent \textbf{I'm getting a ``java.lang.OutOfMemoryError: Java heap space"}\newline
Some operations need a lot of memory. If you simply start SBMLsqueezer, without any
JVM parameters, only 64\,MB of memory are available. Please append the argument
\texttt{-Xmx1024M} to start the application with 1\,GB of main memory. See
\vref{startingTheProgram} for a more detailed description of how to
start the application with additional memory. If possible, you should give the
application 2\,GB of main memory. A minimum of 1\,GB main memory should be
available to the application.\newline

\noindent \textbf{Is an internet connection required to run SBMLsqueezer?}\newline
An internet connection is required for most operations. Many identifier mapping
files and pathway-based visualizations require an active internet connection.
However, if you import your data directly with NCBI Entrez Gene IDs and do not
use the pathway-visualization or GO-enrichment, you should be able to run the
application offline.\newline

\noindent \textbf{Where can I get the latest version?}\newline
Go to \url{http://www.cogsys.cs.uni-tuebingen.de/software/Integrator/}.\newline

\noindent \textbf{Which \Java version must be installed on my computer to launch
SBMLsqueezer?}\newline SBMLsqueezer requires at least \Java 1.6. Please see
\url{http://www.java.com/de/download/} to download the latest \Java version.

\noindent 
\textbf{Why does SBMLsqueezer not start on my Mac with \MacOSX prior to 10.6 Update 3?}\newline
If you try to launch SBMLsqueezer, but the application does
not start and you receive the following error message on the command-line or
\Java console of your Mac, you need to update your \Java installation:
\begin{verbatim}
Exception in thread "AWT-EventQueue-0" java.lang.NoClassDefFoundError:
    com/apple/eawt/AboutHandler
    at java.lang.ClassLoader.defineClass1(Native Method)
    at java.lang.ClassLoader.defineClass(ClassLoader.java:703)
    ...
\end{verbatim}
The interface \texttt{com.apple.eawt.AboutHandler} was introduced to \Java for
\MacOSX 10.6 Update 3. If you have an earlier version of \MacOSX or \Java,
please update your OS or \Java installation. Also see the \MacOSX documentation
about the \texttt{AboutHandler} for more information. On a Mac, you can update
your \Java installation through the Software Update menu item in the main Apple
menu.

\chapter{License}

SBMLsqueezer is free software: you can redistribute it and/or modify
it under the terms of the GNU General Public License as published by
the Free Software Foundation, either version~3 of the License, or
(at your option) any later version.

This program is distributed in the hope that it will be useful,
but \textbf{without any warranty}; without even the implied warranty of
\textbf{merchantability} or \textbf{fitness for a particular purpose}. See the
GNU General Public License for more details.

Each distributed version of SBMLsqueezer should contain a copy of the 
GNU General Public License. If not, please see
\href{http://www.gnu.org/licenses/gpl-3.0-standalone.html}{\nolinkurl{http://www.gnu.org/licenses/}}.

\chapter{Acknowledgments}

This work has been funded by a Marie Curie International Outgoing Fellowship
awarded to Andreas Dr\"ager within the EU 7\textsuperscript{th} Framework
Program for Research and Technological Development (project AMBiCon, 332020) and
by the Federal Ministry of Education and Research (BMBF, Germany) in the
projects Virtual Liver Network (project number 0315756), National Genome
Research Network (NGFN-Plus, project number 01GS08134), and Spher4Sys (grant
number 0315384C).

\section{Core developers}

The following people implemented wide parts of SBMLsqueezer:
\begin{itemize}
\item Andreas Dr\"ager, 
  University of California, San Diego, La Jolla, California, USA and
  University of Tuebingen, Germany
  \href{mailto:andraeger@eng.ucsd.edu}{andraeger@eng.ucsd.edu}
\item Alexander D\"orr, 
  University of Tuebingen, Germany
  \href{mailto:alexander.doerr@uni-tuebingen.de}{alexander.doerr@uni-tuebingen.de}
\item Roland Keller,
  University of Tuebingen, Germany
  \href{mailto:roland.keller@uni-tuebingen.de}{roland.keller@uni-tuebingen.de}
\item Johannes Eichner,
  University of Tuebingen, Germany
  \href{mailto:johannes.eichner@uni-tuebingen.de}{johannes.eichner@uni-tuebingen.de}
\end{itemize}

\section{Principal Investigators}

\begin{itemize}
\item Bernhard \O. Palsson,
  University of California, San Diego, La Jolla, California, USA
  \href{mailto:palsson@ucsd.edu}{palsson@ucsd.edu}
\item Andreas Zell, 
  University of Tuebingen, Germany
  \href{mailto:andreas.zell@uni-tuebingen.de}{andreas.zell@uni-tuebingen.de}
\end{itemize}

\section{Alumni}

During the years, many people contributed to this project.
We are grateful to each contribution, such as source code, advice, proof reading
and much more. In particular, we like to thank our former scientific advisors,
colleagues, and students, who are here listed all together in alphabetical
order:
Meike Aichele,
Hannes Borch,
Nadine Hassis,
Marcel Kronfeld,
Oliver Kohlbacher,
Sarah Rachel M\"uller vom Hagen,
Sebastian Nagel,
Leif J. Pallesen,
Alexander Peltzer,
Julianus Pfeuffer,
Matthias Rall,
Sandra Saliger,
Simon Sch\"afer,
Adrian Schr\"oder,
Jochen Supper,
Dieudonn\'e M. Wouamba,
Michael J. Ziller
